\documentclass{beamer}
%\usepackage[latin1]{inputenc}
\usetheme{Warsaw}
\title[Intro to Python: Week 9]{Introduction  to Python}
\author{Christopher Barker}
\institute{UW Continuing Education / Isilon}
\date{August 29, 2012}

\usepackage{listings}
\usepackage{hyperref}

\begin{document}

% ---------------------------------------------
\begin{frame}
  \titlepage
\end{frame}

% ---------------------------------------------
\begin{frame}
\frametitle{Table of Contents}
%\tableofcontents[currentsection]
  \tableofcontents
\end{frame}


\section{Review/Questions}

% ---------------------------------------------
\begin{frame}[fragile]{Review of Previous Class}

{\Large
\vfill
\begin{itemize}
  \item Decorators
  \item Debugging
  \item Packages and Packaging
\end{itemize}
}

\vfill
{\Large Lightning talk today: Chris}

\end{frame}


% ---------------------------------------------
\begin{frame}[fragile]{Homework review}

\vfill
{\Large Did any of you find a package you couldn't install?}


\vfill
{\Large Did any of you write a \verb|setup.py| for you own code?}

\vfill

{\large Note: I couldn't \verb|'pip install'| pychecker -- I had to download
the tarball and \verb|'python setup.py install'| it.}

\end{frame}

\section{Testing / doctests}

% ---------------------------------------------
\begin{frame}[fragile]{Testing}

{\huge NOTE:}

\vfill
{\LARGE A bit strange to be ``teaching'' testing \\[0.1in]
to a room full of professional testers.}

\vfill

\end{frame} 

% ---------------------------------------------
\begin{frame}[fragile]{Testing}

\vfill
{\Large I don't need to tell you why testing is important}

\vfill
{\Large I'll focus on testing your Python code -- not another system}

\vfill
{\Large Mostly unit testing}

\vfill
{\Large And an introduction to some of the tools}

\vfill

\end{frame} 

% ---------------------------------------------
\begin{frame}[fragile]{Python Testing Taxonomy}

\begin{enumerate}
  \item   Unit Testing Tools
  \item    Mock Testing Tools
  \item    Fuzz Testing Tools
  \item    Web Testing Tools
  \item    Acceptance/Business Logic Testing Tools
  \item    GUI Testing Tools
  \item    Source Code Checking Tools
  \item    Code Coverage Tools
  \item    Continuous Integration Tools
  \item    Automatic Test Runners
  \item    Test Fixtures
  \item    Miscellaneous Python Testing Tools
\end{enumerate}

\vfill
\url{http://wiki.python.org/moin/PythonTestingToolsTaxonomy}

\end{frame} 


% ---------------------------------------------
\begin{frame}[fragile]{Testing}

\vfill
{\LARGE Regression Testing:}

\vfill
{\Large Regression testing is any type of software testing that seeks to
uncover new software bugs, or regressions, in existing functional and
non-functional areas of a system after changes}

\vfill
{\LARGE Unit Testing:}

\vfill
{\Large unit testing is a method by which individual units of source code,
sets of one or more computer program modules together with associated control
data, usage procedures, and operating procedures, are tested to determine if
they are fit for use}

\vfill

\end{frame} 


% ---------------------------------------------
\begin{frame}[fragile]{doctest}

{\LARGEdoctest}

\vfill
{\Large \vspace{0.3in} -- In the stdlib}

\vfill
{\Large \vspace{0.3in} -- Unique to Python?}

\vfill
{\Large \vspace{0.3in} -- Literate programming}

\vfill
{\Large \vspace{0.3in} -- Verify examples in in docs}

\vfill
\url{http://docs.python.org/library/doctest.html}


\end{frame} 

% ---------------------------------------------
\begin{frame}[fragile]{doctest example}

\begin{verbatim}
def get(self):
    """ get() -> return TestClass's associated value.
    >>> x = _TestClass(-42)
    >>> print x.get()
    -42
    """
    return self.val
\end{verbatim}

\vfill
{\Large An exact dump of the command line output}

\vfill
\url{http://docs.python.org/library/doctest.html}
\end{frame} 

% ---------------------------------------------
\begin{frame}[fragile]{doctest uses}

{\large
``
As mentioned in the introduction, doctest has grown to have three primary uses:

\begin{itemize}
    \item Checking examples in docstrings.
    \item Regression testing.
    \item Executable documentation / literate testing.
\end{itemize}

These uses have different requirements, and it is important to distinguish them.
In particular, filling your docstrings with obscure test cases makes for bad
documentation.''
}

\vfill
\url{http://docs.python.org/library/doctest.html}
\end{frame} 

% ---------------------------------------------
\begin{frame}[fragile]{running doctests}

{\large In \verb|__name__ == "__main__"| block:}

\begin{verbatim}
if __name__ == "__main__":
    import doctest
    doctest.testmod()
\end{verbatim}

{\large (Tests the current module) }


\vfill
\url{http://docs.python.org/library/doctest.html}

\end{frame} 

% ---------------------------------------------
\begin{frame}[fragile]{running doctests}

\vfill
{\large In an external file:}

\begin{verbatim}
    doctest.testfile("name_of_test_file.txt")
\end{verbatim}

{\large Tests the docs in the file (ReSt format...) }


\vfill
{\large With a test runner: more on that later}

\vfill
\end{frame} 

% ---------------------------------------------
\begin{frame}[fragile]{running doctests}

\vfill
{\large In a documentation system:}

\vfill
{\LARGE Sphinx:}

\vfill
{\Large Sphinx is a tool that makes it easy to create intelligent and beautiful documentation}

\vfill
{\Large -- Lots of output options: html, pdf, epub, etc, etc...}

\vfill
{\Large -- Can auto-generate docs from docstrings}

\vfill
{\Large -- And run the doctests}

\vfill
\url{http://sphinx.pocoo.org/}
\end{frame} 

% ---------------------------------------------
\begin{frame}[fragile]{Code Checking}

{\LARGE Not Really testing, but...}

\vfill
{\LARGE pychecker}\\
\url{http://pychecker.sourceforge.net/}

\vfill
{\LARGE pylint}\\
\url{http://www.logilab.org/857/}

\vfill
{\LARGE pyflakes}\\
\url{http://pypi.python.org/pypi/pyflakes}

\vfill
{\Large Will help you keep your source code clean}

\end{frame} 

%-------------------------------
\begin{frame}[fragile]{pep8}

{\LARGE\verb|pep8.py|}

\vfill
{\Large tells you when you have code which doesn't follow PEP 8}

\vfill
\begin{verbatim}
pip install pep8
\end{verbatim}

\vfill
\url{http://pypi.python.org/pypi/pep8/}

\end{frame} 

%-------------------------------
\begin{frame}[fragile]{LAB}

{\Large doctests:}

\begin{itemize}
  \item doctest
    \begin{itemize}
       \item Add doctests to the Circle class we've done in the last few classes.
       \item Run it from the \verb|if __name__ == "__main__"| clause\\
             (\verb|code\circle.py|)
    \end{itemize}
  \item Code style
    \begin{itemize}
       \item Try running \verb|pylint, pychecker or pyflakes| on it 
       \item Run pep8 on it
       \item (or your own code)
    \end{itemize}

\end{itemize}

\end{frame}

%-------------------------------
\begin{frame}{Lightning Talk}

{\centering

\vfill
{\LARGE Lightning Talk:  }

\vfill
{\Huge Peter}

\vfill
}
\end{frame}

\section{Unit Testing}

% ---------------------------------------------
\begin{frame}[fragile]{Unit Testing}

{\LARGE Gaining Traction}

\vfill
{\Large You need to test your code when you write it -- why not preserve those tests?}

\vfill
{\Large And allow you to auto-run them later?}

\vfill
{\LARGE Test-Driven development:}\\[0.1in]
{\Large \hspace{0.3in} Write the tests before the code}

\end{frame} 

% ---------------------------------------------
\begin{frame}[fragile]{Unit Testing}

{\LARGE My thoughts:}

\vfill
{\Large Unit testing encourages clean, decoupled design}

\vfill
{\Large If it's hard to write unit tests for -- it's not well designed}

\vfill
{\Large but...}

\vfill
{\Large ``complete'' test coverage is a fantasy}

\end{frame} 


% ---------------------------------------------
\begin{frame}[fragile]{PyUnit}

{\LARGE PyUnit: the stdlib unit testing framework}

\vfill
{\Large \verb|import unittest|}

\vfill
{\Large More or less a port of Junit from Java}

\vfill
{\Large A bit verbose: you have to write classes \& methods}

\end{frame} 

% ---------------------------------------------
\begin{frame}[fragile]{unittest example}

{\small
\begin{verbatim}
import random
import unittest

class TestSequenceFunctions(unittest.TestCase):

    def setUp(self):
        self.seq = range(10)

    def test_shuffle(self):
        # make sure the shuffled sequence does not lose any elements
        random.shuffle(self.seq)
        self.seq.sort()
        self.assertEqual(self.seq, range(10))

        # should raise an exception for an immutable sequence
        self.assertRaises(TypeError, random.shuffle, (1,2,3))
\end{verbatim}
}

\end{frame} 


% ---------------------------------------------
\begin{frame}[fragile]{unittest example (cont)}

{\small
\begin{verbatim}
    def test_choice(self):
        element = random.choice(self.seq)
        self.assertTrue(element in self.seq)

    def test_sample(self):
        with self.assertRaises(ValueError):
            random.sample(self.seq, 20)
        for element in random.sample(self.seq, 5):
            self.assertTrue(element in self.seq)

if __name__ == '__main__':
    unittest.main()
\end{verbatim}
}

\vfill
(\verb|code/unitest_example.py|)

\vfill
\url{http://docs.python.org/library/unittest.html}

\end{frame} 

% ---------------------------------------------
\begin{frame}[fragile]{unittest}

{\Large Lots of good tutorials out there:}

\vfill
{\Large Google: ``python unittest tutorial''}

\vfill
{\Large I learned from this one:}\\[0.1in]
\url{http://www.diveintopython.net/unit_testing/index.html}

\end{frame} 

% ---------------------------------------------
\begin{frame}[fragile]{nose and pytest}

{\Large Due to its Java heritage, unittest is kind of verbose}

\vfill
{\Large also no test discovery}\\
\hspace{0.2in}(though unittest2 does add that...)

\vfill
{\Large So folks invented nose and pytest}

\end{frame} 

\begin{frame}[fragile]{nose}

{\LARGE \verb|nose|}

\vfill
{\Large \hspace{0.2in} Is nicer testing for python}

\vfill
{\Large \hspace{0.2in} nose extends unittest to make testing easier.}

\vfill
\begin{verbatim}
 $ pip install nose

 $ nosetests unittest_example.py 
\end{verbatim}

\vfill
\url{http://nose.readthedocs.org/en/latest/}
\end{frame} 

\begin{frame}[fragile]{nose example}

{\Large The same example -- with nose}

{\small
\begin{verbatim}
import random
import nose.tools

seq = range(10)

def test_shuffle():
    # make sure the shuffled sequence does not lose any elements
    random.shuffle(seq)
    seq.sort()
    assert seq == range(10)

@nose.tools.raises(TypeError)
def test_shuffle_immutable():
    # should raise an exception for an immutable sequence
    random.shuffle( (1,2,3) )
\end{verbatim}
}

\end{frame} 

\begin{frame}[fragile]{nose example (cont) }

{\small
\begin{verbatim}
def test_choice():
    element = random.choice(seq)
    assert (element in seq)

def test_sample():
    for element in random.sample(seq, 5):
        assert element in seq

@nose.tools.raises(ValueError)
def test_sample_too_large():
    random.sample(seq, 20)
\end{verbatim}
}

\vfill
(\verb|code/test_random_nose.py|)

\end{frame} 


\begin{frame}[fragile]{pytest}

{\LARGE \verb|pytest|}

\vfill
{\Large \hspace{0.2in} A mature full-featured testing tool}

\vfill
{\Large \hspace{0.2in} Provides no-boilerplate testing}

\vfill
{\Large \hspace{0.2in} Integrates many common testing methods}

\vfill
\begin{verbatim}
 $ pip install pytest

 $ py.test unittest_example.py 
\end{verbatim}

\vfill
\url{http://pytest.org/latest/}
\end{frame} 

\begin{frame}[fragile]{pytest example}

{\Large The same example -- with pytest}

{\small
\begin{verbatim}
import random
import pytest

seq = range(10)

def test_shuffle():
    # make sure the shuffled sequence does not lose any elements
    random.shuffle(seq)
    seq.sort()
    assert seq == range(10)

def test_shuffle_immutable():
    pytest.raises(TypeError, random.shuffle,  (1,2,3) )
\end{verbatim}
}

\end{frame} 

\begin{frame}[fragile]{pytest example (cont) }

{\small
\begin{verbatim}
def test_choice():
    element = random.choice(seq)
    assert (element in seq)

def test_sample():
    for element in random.sample(seq, 5):
        assert element in seq

def test_sample_too_large():
    with pytest.raises(ValueError):
        random.sample(seq, 20)
\end{verbatim}
}

\vfill
(\verb|code/test_random_pytest.py|)

\end{frame} 

\begin{frame}[fragile]{With}

{\LARGE Context Managers:} {\Large the \verb|with| statement}

\vfill
{\Large A class with \verb|__enter__()| and \verb|__exit__()| methods.}
 
\vfill
{\Large \verb|__enter__()| is run before your block of code}

\vfill
{\Large \verb|__enter__()| is run after your block of code}

\vfill
{\Large Can be used to setup/clenup before and after: open/closing files, db connections, etc}
\end{frame} 

\begin{frame}[fragile]{A Diversion }

{\Large ``PEP 343: the \verb|with| statement''} \\
\hspace{0.2in}  -- A.M. Kuchling

\url{http://docs.python.org/dev/whatsnew/2.6.html#pep-343-the-with-statement}

\vfill
{\Large ``Understanding Python's \verb|with| statement''} \\
\hspace{0.2in}  -- Fredrik Lundh 

\url{http://effbot.org/zone/python-with-statement.htm}

\vfill
{\Large ``The Python \verb|with| Statement by Example''} \\
\hspace{0.2in}  -- Jeff Preshing 

\url{http://preshing.com/20110920/the-python-with-statement-by-example}

\end{frame} 

\begin{frame}[fragile]{Parameterized Tests}

{\Large A whole set of inputs and outputs to test?}

\vfill
{\Large \verb|pytest| has a nice way to do that (so does nose...)}

\begin{verbatim}
import pytest
@pytest.mark.parametrize(("input", "expected"), [
    ("3+5", 8),
    ("2+4", 6),
    ("6*9", 42),
])
def test_eval(input, expected):
    assert eval(input) == expected
\end{verbatim}

\url{http://pytest.org/latest/example/parametrize.html}

\vfill
(\verb|code/test_pytest_parameter.py|)
\end{frame} 

\begin{frame}[fragile]{Test Coverage}

{\LARGE \verb|coverage.py |}

\vfill
{\Large Uses debugging hook to see which lines of code are actually executed
-- plugins exist for most (all?) test runners}

\vfill
{\Large \verb|pip install coverage |}

\vfill
{\Large \verb|nosetests --with-coverage test_codingbat.py|}

\vfill
\url{http://nedbatchelder.com/code/coverage/}
\end{frame}


%-------------------------------
\begin{frame}[fragile]{LAB}

{\Large Unit Testing:}

\begin{itemize}
  \item unittest
    \begin{itemize}
       \item Pick a \url{codingbat.com} example
       \item Write a set of unit tests using \verb|unittest|
         (\verb|code\codingbat.py  codingbat_unittest.py|)
    \end{itemize}
  \item pytest / nose
    \begin{itemize}
       \item Test a \url{codingbat.com} with nose or pytest
       \item Try doing test-driven development
         (\verb|code\test_codingbat.py|)
    \end{itemize}

  \item try running your circle doctest with nose, pytest\\
        (and/or add doctests to your codingbat example)
  \item try running \verb|coverage| on your tests
\end{itemize}

\end{frame}

\section{Profiling}

% ---------------------------------------------
\begin{frame}[fragile]{Performance Testing}

{\Large ``Premature optimization is the root of all evil''}\\[0.1in]
{\large \hspace{0.5in} -- Donald Knuth}

\end{frame} 

% ---------------------------------------------
\begin{frame}[fragile]{Profiling/timing}

\vfill
{\Large You can't optimize your code without knowing where the bottlenecks are.}

\vfill
{\Large Smarter people than me have said they they are almost always wrong
when they try to logically determine where the slow code is. (I know I am)}

\vfill
{\Large ... and how to speed it up}

\end{frame} 

% ---------------------------------------------
\begin{frame}[fragile]{timing}

{\Large The really easy way:}

\begin{verbatim}
import time

start = time.clock()
  ... do_some_stuff ...
print "It took %f seconds to run"%(time.clock - start)
\end{verbatim}

{\Large It works, it's easy, and it gives a gross approximation}

\vfill
(use \verb|time.clock()|, rather than \verb|time.time()|)

\end{frame} 

% ---------------------------------------------
\begin{frame}[fragile]{timing}

{\Large The good way:}

\begin{verbatim}
import timeit

timeit.timeit( statement, setup=some_stuff)

\end{verbatim}

{\Large It's kind of a pain, but gives meaningful results.}

(can also be called on the command line)

\vfill
\url{http://docs.python.org/library/timeit.html}
\vfill
(\verb|code/timing.py|)
\end{frame} 

% ---------------------------------------------
\begin{frame}[fragile]{timing}

{\Large The easy and good way:}

\vfill
{\LARGE \verb|ipython|:}
\begin{verbatim}

In [52]: import timing

In [53]: %timeit timing.primes_stupid(5)
100000 loops, best of 3: 10.9 us per loop

\end{verbatim}

{\Large Takes care of the setup/namespace stuff for you}

\vfill
\url{http://ipython.org/ipython-doc/dev/interactive/tutorial.html}
\end{frame} 


% ---------------------------------------------
\begin{frame}[fragile]{profiling}

{\Large A profiler is a tool that describes the run time performance of a
program, providing a variety of statistics}

\vfill
{\Large Helpful when you don't yet know where your bottlenecks are}

\vfill
{\Large The python profiler}

\begin{verbatim}
python -m cProfile profile_example.py  
\end{verbatim}
{\Large spews some stats}


\vfill
\url{http://docs.python.org/library/profile.html}
\end{frame} 

% ---------------------------------------------
\begin{frame}[fragile]{python profiler}

{\Large What you get:}

\begin{description}
  \item[ncalls] the number of calls.
  \item[tottime] the total time spent in the given function (and excluding time made in calls to sub-functions),
  \item[percall] the quotient of tottime divided by ncalls
  \item[cumtime] the total time spent in this and all subfunctions (from invocation till exit). This figure is accurate even for recursive functions.
  \item[percall] the quotient of cumtime divided by primitive calls
\end{description}

\end{frame} 



%-------------------------------
\begin{frame}{LAB}

{\Large Profiling lab}

\begin{itemize}
  \item
\end{itemize}

\end{frame}



%-------------------------------
\begin{frame}{Wrap up}

\begin{itemize}
  \item something
\end{itemize}

\end{frame}

%-------------------------------
\begin{frame}[fragile]{Next Week:}

\vfill
{\Large Votes are in:}

\vfill
{\Large 4 votes for persistence: 0 for Extending with C}

\vfill
{\LARGE Persistence it is}

\vfill

\end{frame}


%%-------------------------------
%\begin{frame}[fragile]{Homework}
%
%\begin{itemize}
%    \item Make sure you've got your compiler working:
%    \begin{itemize}
%      \item see: (link to class Wiki)
%      \item You should be able to run: \\
%            \verb|week-10/code/setup.py build_ext|
%    \end{itemize}
%    \item Install Cython:
%    \begin{itemize}
%      \item \verb|pip install cython|\\
%        or
%      \item \url{www.cython.org}
%    \end{itemize}
%
%\end{itemize}
%
%\end{frame}


\end{document}

 
