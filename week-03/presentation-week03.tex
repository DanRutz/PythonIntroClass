\documentclass{beamer}
%\usepackage[latin1]{inputenc}
\usetheme{Warsaw}
\title[Intro to Python: Week 2]{Introduction  to Python\\ Exceptions, Unicode, File Processing}
\author{Christopher Barker}
\institute{UW Continuing Education / Isilon}
\date{July 11, 2012}

\usepackage{listings}
\usepackage{hyperref}

\begin{document}

% ---------------------------------------------
\begin{frame}
  \titlepage
\end{frame}

% ---------------------------------------------
\begin{frame}
\frametitle{Table of Contents}
%\tableofcontents[currentsection]
  \tableofcontents
\end{frame}

\section{Review/Questions}


%-------------------------------
\begin{frame}{Last Class}

\vfill
(Bill and Joshua -- ready for your lightning talk?)

\vfill
{\Large Lab from end of last class?  }

\vfill
\end{frame}


%-------------------------------
\begin{frame}[fragile]{LAB}

\verb| def count_them(letter): |
\begin{itemize}
  \item prompts the user to input a letter
  \item counts the number of times the given letter is input
  \item prompts the user for another letter
  \item continues until the user inputs "x"
  \item returns the count of the letter input
\end{itemize}

\verb| def count_letter_in_string(string, letter): |
\begin{itemize}
  \item counts the number of instances of the letter in the string
  \item ends when a period is encountered
  \item if no period is encountered -- prints "hey, there was no period!"
\end{itemize}
\end{frame}


% ---------------------------------------------
\begin{frame}{Questions?}

{\Large Any Questions about:
\begin{itemize}
  \item Last class ?
  \item Reading ?
  \item Homework ?
\end{itemize}
}
\end{frame}


% header
% ---------------------------------------------
\begin{frame}{Homework review}

  {\Large Homework notes }

\end{frame}

\section {Notes on Reading}

%-------------------------------
\begin{frame}[fragile]{subprocesses}

{\Large Subprocesses}
\begin{verbatim}
#easy:  
os.popen('ls').read()

#even easier:
os.system('ls')

# but for anything more complicated:
pipe = \
  subprocess.Popen("ls", stdout=subprocess.PIPE).stdout 
\end{verbatim}

\end{frame}


%-------------------------------
\begin{frame}[fragile]{reload}

{\Large module importing and reloading}
\begin{verbatim}
In [190]: import module_reload

In [191]: module_reload.print_something()
I'm printing something

# change it...
In [196]: reload(module_reload)
Out[196]: <module 'module_reload' from 'module_reload.py'>

In [193]: module_reload.print_something()
I'm printing something else
\end{verbatim}

\end{frame}

% ---------------------------------------------
\begin{frame}[fragile]{Module Reloading}

\begin{verbatim}
In [194]: from module_reload import this

# change it...

In [196]: reload(module_reload)
Out[196]: <module 'module_reload' from 'module_reload.py'>

In [197]: module_reload.this
Out[197]: 'this2'

In [198]: this
Out[198]: 'this'
\end{verbatim}

\end{frame}

% ---------------------------------------------
\begin{frame}[fragile]{repr vs. str}

{\Large \verb|repr()| vs \verb|str()| }

\begin{verbatim}
In [200]: s = "a string\nwith a newline"

In [203]: print str(s)
a string
with a newline

In [204]: print repr(s)
'a string\nwith a newline'

\end{verbatim}

\end{frame}

% ---------------------------------------------
\begin{frame}[fragile]{repr vs. str}

{\Large \verb| eval(repr(something)) == something | }

\begin{verbatim}

In [205]: s2 = eval(repr(s))

In [206]: s2
Out[206]: 'a string\nwith a newline'
\end{verbatim}

\end{frame}


\section{String Formatting, etc}

% ---------------------------------------------
\begin{frame}[fragile]{Strings}

{\Large A string literal creates a string type}

\begin{verbatim}
"this is a string"
\end{verbatim}

{\Large Can also use \verb|str()|}

\begin{verbatim}
In [256]: str(34)
Out[256]: '34'
\end{verbatim}
{\Large or "back ticks"}
\begin{verbatim}
In [258]: `34`
Out[258]: '34'
\end{verbatim}
(demo)
\end{frame} 

% ---------------------------------------------
\begin{frame}[fragile]{The String Type}

{\Large Lots of nifty methods:}

\begin{verbatim}
s.lower()
s.upper()
     ...
s.capitalize()
s.swapcase()
s.title()
\end{verbatim}

\url{http://docs.python.org/library/stdtypes.html#index-23}

\end{frame} 

% ---------------------------------------------
\begin{frame}[fragile]{The String Type}

{\Large Lots of nifty methods:}

\begin{verbatim}
x in s
s.startswith(x)
s.endswith
...
s.index(x)
s.find(x)
s.rfind(x)
\end{verbatim}

\url{http://docs.python.org/library/stdtypes.html#index-23}

\end{frame} 

% ---------------------------------------------
\begin{frame}[fragile]{The String Type}

{\Large Lots of nifty methods:}

\begin{verbatim}
s.split()
s.join(list)
...
s.splitlines()
\end{verbatim}

\url{http://docs.python.org/library/stdtypes.html#index-23}
\vfill

\end{frame} 

% ---------------------------------------------
\begin{frame}[fragile]{Joining Strings}

{\Large The Join Method:}

\vfill
\begin{verbatim}
In [289]: t = ("some", "words","to","join")

In [290]: " ".join(t)
Out[290]: 'some words to join'

In [291]: ",".join(t)
Out[291]: 'some,words,to,join'

In [292]: "".join(t)
Out[292]: 'somewordstojoin'
\end{verbatim}

\vfill
(demo -- join)

\end{frame} 

% ---------------------------------------------
\begin{frame}[fragile]{The string module}

{\Large Lots of handy constants, etc.}
\begin{verbatim}
string.ascii_letters
string.ascii_lowercase 
string.ascii_uppercase  
string.letters
string.hexdigits 
string.whitespace
string.printable
string.digits
string.punctuation      
\end{verbatim}

\vfill
(and the string methods -- legacy)
\end{frame} 

% ---------------------------------------------
\begin{frame}[fragile]{String Literals}

{\Large Common Escape Sequences}
\vfill
\begin{verbatim}
\\ 	Backslash (\) 	
\a 	ASCII Bell (BEL) 	
\b 	ASCII Backspace (BS) 	
\n 	ASCII Linefeed (LF) 	
\r 	ASCII Carriage Return (CR) 	
\t 	ASCII Horizontal Tab (TAB) 	
\ooo 	Character with octal value ooo 
\xhh 	Character with hex value hh
\end{verbatim}
(\url{http://docs.python.org/release/2.5.2/ref/strings.html})
\end{frame} 

% ---------------------------------------------
\begin{frame}[fragile]{Raw Strings}

{\Large Escape Sequences Ignored}
\vfill
\begin{verbatim}
In [408]: print "this\nthat"
this
that
In [409]: print r"this\nthat"
this\nthat
\end{verbatim}

{\Large Gotcha:}
\begin{verbatim}
In [415]: r"\"
SyntaxError: EOL while scanning string literal
\end{verbatim}

\vfill
(handy for regex, windows paths...)
\end{frame} 


% ---------------------------------------------
\begin{frame}[fragile]{Building Strings}

{\Large Please don't do this:

\vfill
\begin{verbatim}
'Hello ' + name + '!'
\end{verbatim}
}
\vfill
(much)

\end{frame} 

% ---------------------------------------------
\begin{frame}[fragile]{Building Strings}

{\Large Do this instead:

\vfill
\begin{verbatim}
'Hello %s!' % name
\end{verbatim}

\vfill
much faster and safer:

\vfill
easier to modify as code gets complicated
}

\vfill
\url{http://docs.python.org/library/stdtypes.html#string-formatting-operations}
\end{frame} 





% ---------------------------------------------
\begin{frame}[fragile]{String Formatting}

{\Large The string format operator: \%}

\begin{verbatim}
In [261]: "an integer is: %i"%34
Out[261]: 'an integer is: 34'

In [262]: "a floating point is: %f"%34.5
Out[262]: 'a floating point is: 34.500000'

In [263]: "a string is: %s"%"anything"
Out[263]: 'a string is: anything'
\end{verbatim}

\end{frame} 

% ---------------------------------------------
\begin{frame}[fragile]{String Formatting}

{\Large multiple arguments:}

\begin{verbatim}
In [264]: "the number %s is %i"%('five', 5)
Out[264]: 'the number five is 5'

In [266]: "the first 3 numbers are: %i, %i, %i"%(1,2,3)
Out[266]: 'the first 3 numbers are: 1, 2, 3'

\end{verbatim}

\end{frame} 

%-------------------------------
\begin{frame}[fragile]{String formatting}

{\Large Gotcha}

\begin{verbatim}
In [127]: "this is a string with %i formatting item"%1 
Out[127]: 'this is a string with 1 formatting item'

In [128]: "string with %i formatting %s: "%2, "items" 
TypeError: not enough arguments for format string

# Done right:
In [131]: "string with %i formatting %s"%(2, "items")
Out[131]: 'string with 2 formatting items'

In [132]: "string with %i formatting item"%(1,)
Out[132]: 'string with 1 formatting item' 
\end{verbatim}

\end{frame}

%-------------------------------
\begin{frame}[fragile]{String formatting}

{\Large Named arguments}

\begin{verbatim}
'Hello %(name)s!'%{'name':'Joe'}
'Hello Joe!'

'Hello %(name)s, how are you, %(name)s!' %{'name':'Joe'}
'Hello Joe, how are you, Joe!'
\end{verbatim}
\vfill
{\Large That last bit is a dictionary (next week) }

\end{frame}

%-------------------------------
\begin{frame}[fragile]{Advanced Formatting}

{\Large The format method}

\begin{verbatim}
In [14]: 'Hello {0} {1}!'.format('Joe', 'Barnes')
Out[14]: 'Hello Joe Barnes!'

In [12]: 'Hello {name}!'.format(name='Joe')
Out[12]: 'Hello Joe!'
\end{verbatim}
\vfill
{\Large pick one (probably regular string formatting): \\
  -- get comfy with it }

\end{frame}


%-------------------------------
\begin{frame}[fragile]{LAB}

{\Large Fun with strings}

\begin{itemize}
  \item Rewrite: \verb| the first 3 numbers are: %i, %i, %i"%(1,2,3)| \\
        for an arbitrary number of numbers...
  \item write a format string that will take:\\
        \verb|( 2, 123.4567, 10000)| \\
        and produce: \\
        \verb|'file_002 :   123.46, 1e+04'|
  \item Write a (really simple) mail merge program
  \item ROT13 -- see next slide
\end{itemize}

\end{frame}

%-------------------------------
\begin{frame}[fragile]{LAB}

\vfill
\LargeROT13 encryption

\vfill
Applying ROT13 to a piece of text merely requires examining its alphabetic
characters and replacing each one by the letter 13 places further along in
the alphabet, wrapping back to the beginning if necessary

\begin{itemize}
  \item Implement rot13 decoding 
  \item  decode this message: \\
     \hspace{0.5in} Zntargvp sebz bhgfvqr arne pbeare \\
     \hspace{0.5in} (from a geo-caching hint)
\end{itemize}

\end{frame}

%-------------------------------
\begin{frame}{Lightning Talk}

{\center

\LARGE Lighting Talk:
\vfill
Bill
\vfill

}

\end{frame}

\section{File Reading and Writing}

%-------------------------------
\begin{frame}[fragile]{Files}

{\Large Text Files}

\begin{verbatim}
f = open('secrets.txt')
secret_data = f.read()
f.close()
\end{verbatim}

{\Large \verb|secret_data| is a string}

\vfill
(can also use \verb|file()| -- \verb|open()| is preferred)
\end{frame}

%-------------------------------
\begin{frame}[fragile]{Files}

{\Large Binary Files}

\begin{verbatim}
f = open('secrets.txt', 'rb')
secret_data = f.read()
f.close()
\end{verbatim}

{\Large \verb|secret_data| is still a string \\[.1in]
(with arbitrary bytes in it)}
\vfill
(See the \verb|struct| module to unpack binary data )
\end{frame}

%-------------------------------
\begin{frame}[fragile]{Files}

{\Large File Opening Modes}
\vfill
\begin{verbatim}
f = open('secrets.txt', [mode])

'r', 'w', 'a'
'rb', 'wb', 'ab'
r+, w+, a+
r+b, w+b, a+b
U
U+
\end{verbatim}
\vfill
{\Large Gotcha -- w mode always clears the file}
\end{frame}

%-------------------------------
\begin{frame}[fragile]{Text File Notes}

{\Large Text is default}
\begin{itemize}
  \item Newlines are translated: \verb|\r\n -> \n|
  \item   -- reading and writing!
  \item Use *nux-style in your code: \verb|\n|
  \item Open text files with \verb|'U'| "Universal" flag
\end{itemize}

\vfill
{\Large Gotcha:}
\begin{itemize}
  \item  no difference between text and binary on *nix\\
  \begin{itemize}
    \item breaks on Windows
  \end{itemize}
\end{itemize}

\end{frame}

%-------------------------------
\begin{frame}[fragile]{File Reading}

{\Large Reading Part of a file}

\begin{verbatim}
header_size = 4096

f = open('secrets.txt')
secret_data = f.read(header_size)
f.close()
\end{verbatim}

\end{frame}

%-------------------------------
\begin{frame}[fragile]{File Reading}

{\Large Common Idioms}

\begin{verbatim}
for line in open('secrets.txt'):
    print line
\end{verbatim}

\begin{verbatim}
f = open('secrets.txt')
while True:
    line = f.readline()
    if not line: 
        break
    do_something_with_line()
\end{verbatim}

\end{frame}

%-------------------------------
\begin{frame}[fragile]{File Writing}

\begin{verbatim}

outfile = open('output.txt')

for i in range(10):
    outfile.write("this is line: %i\n"%i)

\end{verbatim}

\end{frame}

%-------------------------------
\begin{frame}[fragile]{File Methods}

{\Large Commonly Used Methods}
\begin{verbatim}

f.read() f.readline()  f.readlines() 

f.write(str) f.writelines(seq)
 
f.seek(offset)   f.tell()

f.flush()           
           
f.close() 
\end{verbatim}

\end{frame}

%-------------------------------
\begin{frame}[fragile]{File Like Objects}

{\Large File-like objects }
\vfill
{\large Many classes implement the file interface:}
\vfill
\begin{itemize}
  \item loggers
  \item \verb|sys.stdout|
  \item \verb|urllib.open()|
  \item pipes, subprocesses
  \item StringIO
\end{itemize}

\url{http://docs.python.org/library/stdtypes.html#bltin-­‐file-­‐objects}
\end{frame}

%-------------------------------
\begin{frame}[fragile]{StringIO}

{\Large StringIO }
\vfill
\begin{verbatim}
In [417]: import StringIO
In [420]: f = StringIO.StringIO()

In [421]: f.write("somestuff")

In [422]: f.seek(0)

In [423]: f.read()
Out[423]: 'somestuff'
\end{verbatim}

{\Large handy for testing}
\end{frame}


\section{Unicode}

\begin{frame}[fragile]{Unicode}

{\Large I hope you all read this:}

\vfill
{\Large
\centering
The Absolute Minimum Every Software Developer Absolutely,
Positively Must Know About Unicode and Character Sets (No Excuses!)

}

\vfill
\url{http://www.joelonsoftware.com/articles/Unicode.html}

\vfill
{\Large If not -- go read it!}

\end{frame}

\begin{frame}[fragile]{Unicode}

{\Large
\vfill

Everything is Bytes

\vfill
If it's on disk or on a network, it's bytes

\vfill
Python provides some abstractions to make it easier to deal with bytes

\vfill
}

\end{frame}

\begin{frame}[fragile]{Unicode}

{\Large
\vfill

Unicode is a biggie

\vfill
strings vs unicode 
}

{\large (\verb|str()| vs. \verb|unicode()| ) }

\vfill
{\Large python 2.x vs 3.x}


\vfill
(actually, dealing with numbers rather than bytes is big -- but we take that for granted)

\end{frame}

\begin{frame}[fragile]{Unicode}

{\Large
\vfill
Strings are sequences of bytes

\vfill
Unicode strings are sequences of platonic characters

\vfill
Platonic characters cannot be written to disk or network!
}
\vfill
(ANSI -- one character == one byte -- so easy!)
\end{frame}

\begin{frame}[fragile]{Unicode}

{\Large
\vfill
the \verb|unicode| object lets you work with characters

\vfill
encoding is converting from a uncode object to bytes

\vfill
decoding is converting from bytes to a unicode object

}
\vfill
\end{frame}

\begin{frame}[fragile]{Unicode}

\begin{verbatim}
import codecs
ord()
chr()
unichr()
str()
unicode()
encode()
decode()
\end{verbatim}
\end{frame}

\begin{frame}[fragile]{Unicode Literals}


{\Large 1) Use unicode in your source files:}

\begin{verbatim}
# -*- coding: utf-8 -*-
\end{verbatim}

\vfill
{\Large 2) escape the unicode characters}

\begin{verbatim}
print u"The integral sign: \u222B"
print u"The integral sign: \N{integral}"
\end{verbatim}

{\large lots of tables of code points online:}

\url{http://inamidst.com/stuff/unidata/}

\end{frame}


%---------------------------------
\begin{frame}[fragile]{Unicode}

{\Large
Use unicode objects in all your code

\vfill
decode on input

\vfill
encode on output

\vfill
Many packages do this for you\\
\hspace{0.25in} (XML processing, databases, ...)

\vfill
Gotcha:\\
\hspace{0.25in} Python has a default encoding (usually ascii)
}
\end{frame}

\begin{frame}[fragile]{Unicode}

{\Large Python Docs Unicode HowTo:}

\url{http://docs.python.org/howto/unicode.html}

\vfill
``Reading Unicode from a file is therefore simple:''

\begin{verbatim}
import codecs
f = codecs.open('unicode.rst', encoding='utf-8')
for line in f:
    print repr(line)
\end{verbatim}

\end{frame}

%-------------------------------
\begin{frame}[fragile]{LAB}

{\Large unicode exercises}
\begin{itemize}
  \item Find some nifty non-ascii characters you might use.\\
        Create a unicode object with them in two different ways.
  \item In the "code" dir for this week, there are two files:\\
        \verb|text.utf16| \\
        \verb|text.utf16| \\
        read the contents into unicode objects
  \item write some of the text from the first exercise to file.
  \item read that file back in.
\end{itemize}

\vfill
(online unicode reference: \url{http://inamidst.com/stuff/unidata/})
\end{frame}

%-------------------------------
\begin{frame}{Lightning Talk}

{\center

\LARGE Lighting Talk:
\vfill
Joshua
\vfill

}
\end{frame}

\section{Exceptions}

\begin{frame}[fragile]{Exceptions}

{\Large Another Branching structure:}
\vfill
\begin{verbatim}
try:
    do_something()
    f = open('missing.txt')
    process(f)   # never called if file missing
except IOError:
    print "couldn't open missing.txt"
\end{verbatim}
\vfill
\end{frame}

\begin{frame}[fragile]{Exceptions}

{\Large Never Do this:}
\vfill
\begin{verbatim}
try:
    do_something()
    f = open('missing.txt')
    process(f)   # never called if file missing
except:
    print "couldn't open missing.txt"
\end{verbatim}
\vfill
\end{frame}

\begin{frame}[fragile]{Exceptions}

{\Large Use Exceptions, rather than your own tests\\
 -- Don't do this:}
\vfill
\begin{verbatim}
do_something()
if os.path.exists('missing.txt'):
    f = open('missing.txt')
    process(f)   # never called if file missing
\end{verbatim}
\vfill
it will almost always work -- but the almost will drive you crazy
\end{frame}

\begin{frame}[fragile]{Exceptions}

{\centering

{\Large "easier to ask forgiveness than permission"
\vfill
\hfill -- Grace Hopper
}

\vfill
\url{http://www.youtube.com/watch?v=AZDWveIdqjY}
\end{frame}

\begin{frame}[fragile]{Exceptions}

\vfill
{\Large 
For simple scripts, let exceptions happen\\
\vfill

Only handle the exception if the code can and will do something about it
}
\vfill
(much better debugging info when an error does occur)
\end{frame}


\begin{frame}[fragile]{Exceptions -- finally }

\vfill
\begin{verbatim}
try:
    do_something()
    f = open('missing.txt')
    process(f)   # never called if file missing
except IOError:
    print "couldn't open missing.txt"
finally:
    do_some_clean-up
\end{verbatim}
\vfill
{\Large the \verb|finally:| clause will always run}
\end{frame}

\begin{frame}[fragile]{Exceptions -- else }

\vfill
\begin{verbatim}
try:
    do_something()
    f = open('missing.txt')
except IOError:
    print "couldn't open missing.txt"
else:
    process(f) # only called if there was no exception
\end{verbatim}
\vfill
{\Large Advantage:

you know where the Exception came from}
\end{frame}

%--------------------------------------------
\begin{frame}[fragile]{Exceptions -- using them }

\vfill
\begin{verbatim}
try:
    do_something()
    f = open('missing.txt')
except IOError as the_error:
    print the_error
    the_error.extra_info = "some more information"
    raise
\end{verbatim}

{\Large Particularly useful if you catch more than one exception:}

\begin{verbatim}
except (IOError, BufferError, OSError) as the_error:
    do_something_with (the_error)
\end{verbatim}

\end{frame}


\begin{frame}[fragile]{Raising Exceptions }

\begin{verbatim}
def divide(a,b):
    if b == 0:
        raise ZeroDivisionError("b can not be zero")
    else:
        return a / b
\end{verbatim}
\vfill
{\Large when you call it: }
\vfill
\begin{verbatim}
In [515]: divide (12,0)

ZeroDivisionError: b can not be zero
\end{verbatim}

\end{frame}



\begin{frame}[fragile]{Built in Exceptions}

{\Large You can create your own custom exceptions}

{\Large But...}

\begin{verbatim}
exp = \
 [name for name in dir(__builtin__) if "Error" in name]

len(exp)
32
\end{verbatim}

{\Large For the most part, you can/should use a built in one}

\end{frame}


%
%%-------------------------------
%\begin{frame}{LAB}
%{\Large Exceptions Lab}
%\begin{itemize}
%  \item
%\end{itemize}
%
%\end{frame}

%-------------------------------
\begin{frame}{Lightning Talk}

{\center

\LARGE Lighting Talk:
\vfill
Joshua
\vfill

}
\end{frame}

\section{Paths and Directories}

% ----------------------------------
\begin{frame}[fragile]{Paths}

{\Large Relative paths:}

\begin{verbatim}
secret.txt
./secret.txt
\end{verbatim}

{\Large Absolute paths:}
\begin{verbatim}
/home/chris/secret.txt
\end{verbatim}

{\Large Either work with \verb|open()|, etc.}

\vfill
(working directory only makes sense with command-line programs...)
\end{frame}

% ----------------------------------
\begin{frame}[fragile]{os.path}

\begin{verbatim}
os.getcwd() -- os.getcwdu()
chdir(path)

os.path.abspath()
os.path.relpath()
\end{verbatim}

\end{frame}

% ----------------------------------
\begin{frame}[fragile]{os.path}

\vfill
\begin{verbatim}
os.path.split()
os.path.splitext()
os.path.basename()
os.path.dirname()
os.path.join()
\end{verbatim}

\vfill
(all platform independent)

\end{frame}


% ----------------------------------
\begin{frame}[fragile]{directories}

\vfill
\begin{verbatim}
os.listdir()
os.mkdir()

os.walk()

\end{verbatim}

\vfill
(higher level stuff in \verb|shutil| module)

\end{frame}

%-------------------------------
\begin{frame}{LAB}

\begin{itemize}
  \item write a program which prints the full path to all files in the current
        directory, one per line
  \item write a program which copies a file from a source, to a
        destination (without using shutil, or the OS copy command)
   \item update mail-merge from the previous lab to write output
         to individual files on disk
\end{itemize}

\end{frame}

%-------------------------------
\begin{frame}{Homework}

\begin{itemize}
  \item TP: Chapters -- 10, 11, 12, 13
  \item Finish (or re-factor) the Labs you didn't finish in class.
  \item CodingBat - 12 more string & list problems
  \item Write a script which does something useful (to you) and reads & writes
        files. Very, very small scope is good. something useful at work would
        be great, but no job secrets!
\end{itemize}

\end{frame}


\end{document}

 
