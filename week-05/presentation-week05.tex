\documentclass{beamer}
%\usepackage[latin1]{inputenc}
\usetheme{Warsaw}
\title[Week 5: Building a Web Server from Scratch]{Introduction to Python}
\author{Christopher Barker}
\institute{UW Continuing Education / Isilon}
\date{July 25, 2012}

\usepackage{listings}
\usepackage{hyperref}

\begin{document}

% ---------------------------------------------
\begin{frame}
  \titlepage
\end{frame}

% ---------------------------------------------
\begin{frame}
\frametitle{Table of Contents}
%\tableofcontents[currentsection]
  \tableofcontents
\end{frame}


\section{Review/Questions}

% ---------------------------------------------
\begin{frame}{Review of Previous Class}

\begin{itemize}
  \item Keyword arguments/parameters
  \item Lists
  \item Dictionaries
  \item Sets
\end{itemize}

\end{frame}


% header
% ---------------------------------------------
\begin{frame}{Homework review}

  {\Large Homework notes }

\end{frame}

\begin{frame}{Class Structure}

{\Large
This class is different -- more a tutorial than a class: lots of coding.

\vfill
We're going to run through building a really basic HTTP server from the ground up.

\vfill
We'll see how far we get.
}

\vfill
Note: I'm no expert -- I'm learning along with you...
\end{frame}

\section{The Protocols}

% ---------------------------------------------
\begin{frame}[fragile]{Sockets}

{\Large ``Socket'' at either end of a pathway: client and server can be
"plugged in" to communicate}

{\Large Five pieces of data to uniquely identify a connection}

\begin{itemize}
  \item Transport protocol (UDP, TCP) (we'll use TCP)
  \item remote IP address
  \item Remote port number
  \item Local IP address
  \item Local port number 
\end{itemize}

\vfill
(use localhost on both ends for this class...)
\end{frame} 

% ---------------------------------------------
\begin{frame}[fragile]{HTTP}

{\LARGE {\bf H}yper{\bf T}ext {\bf T}ransfer {\bf P}rotocol}

 {\Large Client-Server: }
\begin{itemize}
  \item requests
  \item responses
\end{itemize}

{\Large Each has:}

\begin{itemize}
  \item Method specification (request)
  \item Status line (response)
  \item  Headers (RFC 822-compliant)\\[0.1in]
  (optionally)
  \item Entity headers
  \item Blank line
  \item Entity body
\end{itemize}
 
\vfill
 ( RFC 2616 )

\end{frame} 

%-------------------------------
\begin{frame}{HTTP Requests}

{\Large Request Methods  }

\begin{itemize}
  \item GET -- Request a URI content
  \item HEAD -- GET headers only
  \item POST -- PUT save URI content
  \item PUT  -- POST Request URI content, with entity transfer to server
\end{itemize}

\vfill
There are four others -- but these are the ones most used
\end{frame}

%-------------------------------
\begin{frame}[fragile]{HTTP request}

{\Large Example HTTP GET request}
\begin{verbatim}
GET /a_file HTTP/1.1
Host: localhost:55555
User-Agent: Mozilla/5.0 (Macintosh; Intel Mac OS X 10.7; rv:12.0) Gecko/20100101 Firefox/12.0
Accept: text/html,application/xhtml+xml,application/xml;q=0.9,*/*;q=0.8
Accept-Language: en-us,en;q=0.5
Accept-Encoding: gzip, deflate
Connection: keep-alive
\end{verbatim}

\end{frame}

%-------------------------------
\begin{frame}{HTTP Responses}

{\Large Response Codes}

\begin{itemize}
\item 200 OK
\item 404 Not Found
\item 301 Moved Permanently
\item 302 Moved Temporarily
\item 303 See Other (HTTP 1.1 only)
\item 500 Server Error
\end{itemize}

\vfill
There are four others -- but these are the ones most used
\end{frame}

%-------------------------------
\begin{frame}[fragile]{HTTP Response header}

\begin{verbatim}
HTTP/1.0 200 OK
Date: Fri, 31 Dec 1999 23:59:59 GMT
Content-Type: text/html
Content-Length: 1354

<html>
<body>
<h1>Happy New Millennium!</h1>
(more file contents)
... </body> </html>
\end{verbatim}

\vfill
Blank line between header and body critical!\\
 \hspace{0.25in} (\verb|\r\n| linefeeds)
\end{frame}

%-------------------------------
\begin{frame}[fragile]{HTTP Response header}

{\Large Header-Name: value}

\vfill
{\Large Quick reference to HTTP headers:}

\vfill
\url{http://www.cs.tut.fi/~jkorpela/http.html}
\end{frame}

%-------------------------------
\begin{frame}[fragile]{HTTP Response header}

{\Large body data:

\vfill
\verb| Content-Type: : xyz 

\vfill
Mime types we might want:
}

\begin{itemize}
  \item \verb|text/plain|
  \item \verb|text/html|
  \item \verb|image/png|
  \item \verb|image/jpeg|
\end{itemize}

\vfill
\url{http://www.webmaster-toolkit.com/mime-types.shtml}
\end{frame}

%-------------------------------
\begin{frame}[fragile]{Debugging}

{\LargeDebugging Tools}

\begin{itemize}
  \item windows:\\
    \url{http://www.fiddler2.com/fiddler2/}
  \item windows \& mac:\\
    \url{http://www.charlesproxy.com/}
  \item Firefox:\\ 
    \url{http://getfirebug.com/}
  \item Safari, Chrome and IE: built in
\end{itemize}

\end{frame}


%%-------------------------------
\begin{frame}[fragile]{Building an HTTP Server}

\vfill
{\Large We've got everything we need to know to build a simple server}

\vfill
(GET only for now...)

\vfill
{\Large Build an HTTP server that can serve up the files in: \\[0.1in]
\verb|week-05\code\web|}\\

\vfill

\end{frame}

%-------------------------------
\begin{frame}{Building an HTTP Server}

{\Large Incremental Development:}
\begin{enumerate}
  \item A socket server that can receive a request (and print that request to the console)
  \item Server returns a simple reply
  \item Server returns a properly formatted HTML reply
  \item Server returns a directory listing
  \item Server returns the file asked for
  \item Server returns multiple file types
\end{enumerate}

\end{frame}



%-------------------------------
\begin{frame}[fragile]{Python Socket Module}

{\Large }
\begin{verbatim}
s = socket.socket(socket.AF_INET, socket.SOCK_STREAM)
\end{verbatim}
\verb| AF_INET | : Internet Family of Protocols

\verb| SOCK_STREAM | : TCP

\vfill
{\Large Set some options:}
\begin{verbatim}
s.setsockopt(socket.SOL_SOCKET, socket.SO_REUSEADDR, 1)  
\end{verbatim}
\verb| SOL_SOCKET |: ???

\verb| SO_REUSEADDR |: re-use the address -- so the OS won't reserve it


\end{frame}



\end{document}

 
