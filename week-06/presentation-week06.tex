\documentclass{beamer}
%\usepackage[latin1]{inputenc}
\usetheme{Warsaw}
\title[Intro to Python: Week 6]{Introduction  to Python:\\Object Oriented Programming}
\author{Christopher Barker}
\institute{UW Continuing Education / Isilon}
\date{August 01, 2012}

\usepackage{listings}
\usepackage{hyperref}

\begin{document}

% ---------------------------------------------
\begin{frame}
  \titlepage
\end{frame}

% ---------------------------------------------
\begin{frame}
\frametitle{Table of Contents}
%\tableofcontents[currentsection]
  \tableofcontents
\end{frame}

%-------------------------------
\begin{frame}{Lightning Talks}

{\centering

\vfill
{\LARGE Lightning Talks today:  }

\vfill
{\Huge Brett and Matt}

\vfill
}
\end{frame}


\section{Review/Questions}

% ---------------------------------------------
\begin{frame}[fragile]{Review of Previous Class}

\begin{itemize}
  \item Built an HTTP server
  \item Very basics of sockets
  \item Basics of HTTP protocol
  \item Got a server working (even proto-CGI!)
\end{itemize}
\vfill
review of my \verb|http_serve8.py|
\end{frame}


\section{Object Oriented Programming}

% ---------------------------------------------
\begin{frame}[fragile]{Object Oriented Programming}

\vfill
 {\Large More about Python implementation than OO design/strengths/weaknesses}

\vfill
{\Large One reason for this:\\
Folks can't even agree on what OO ``really`` means}

\vfill
The Quarks of Object-Oriented Development - Deborah J. Armstrong:\\
\url{http://agp.hx0.ru/oop/quarks.pdf}

\end{frame} 

% ---------------------------------------------
\begin{frame}[fragile]{Object Oriented Programming}

\vfill
 {\LARGE Is Python a ``True'' Object-Oriented Language?}

\vfill
{\Large (Doesn't support full encapsulation, doesn't require objects, etc...)}

\end{frame} 

% ---------------------------------------------
\begin{frame}[fragile]{Object Oriented Programming}

\vfill
 {\LARGE I don't Care!}

\vfill
{\Large Good software design is about code-reuse, clean separation of concerns,
refactorability, testability, etc...}

\vfill
{\Large OO can help with all that, but:
\begin{itemize}
  \item it doesn't guarantee it
  \item it can get in the way
\end{itemize}
}

\end{frame} 

% ---------------------------------------------
\begin{frame}[fragile]{Object Oriented Programming}

\vfill
 {\LARGE Python is a Dynamic Language}

\vfill
{\Large That clashes with ``pure'' OO}

\vfill
{\Large Think in terms of what makes sense for you project

 -- not any one paradigm of software design.
}


\end{frame} 

% ---------------------------------------------
\begin{frame}[fragile]{Object Oriented Programming}

\vfill
 {\LARGE OO for this class:}

\vfill
{\Large 
``Objects can be thought of as wrapping their data \\[.03in]
within a set of functions designed to ensure that \\[.03in]
the data are used appropriately, and to assist in \\[.03in]
that use``
}

\vfill
{\small
\url{http://en.wikipedia.org/wiki/Object-oriented_programming}
}

\end{frame} 

% ---------------------------------------------
\begin{frame}[fragile]{Object Oriented Programming}

\vfill
{\LARGE Even simpler:}

\vfill
{\Large 
Objects are data and the functions that act on them in one place.
}

\vfill
{\Large 
In Python: just another namespace.
}
\end{frame} 

% ---------------------------------------------
\begin{frame}[fragile]{Object Oriented Programming}

\vfill
{\LARGE The OO buzzwords:

\vfill
\begin{itemize}
  \item data abstraction
  \item encapsulation
  \item messaging
  \item modularity
  \item polymorphism
  \item inheritance
\end{itemize}
}
\end{frame} 

% ---------------------------------------------
\begin{frame}[fragile]{Object Oriented Programming}

\vfill
{\LARGE You can do OO in C}

(see the GTK+ project)

\vfill
{\Large 
``OO languages'' give you some handy tools to make it easier (and safer).
}

\vfill
\begin{itemize}
  \item polymorphism (duck typing gives you this anyway)
  \item inheritance
\end{itemize}

\end{frame} 

% ---------------------------------------------
\begin{frame}[fragile]{Object Oriented Programming}

\vfill
{\LARGE OO is the dominant model for the past couple decades

\vfill
You will need to use it:

\vfill
-- It's a good idea for a lot of problems

\vfill
-- You'll need to work with OO packages
}
\end{frame} 

% ---------------------------------------------
\begin{frame}[fragile]{Object Oriented Programming}

\vfill
{\LARGE Some definitions}

\begin{description}
  \item[class] A category of objects: particular data and behavior: A circle (same as a type in python)
  \item[instance] A particular object of a class: a specific circle
  \item[object] the general case of a instance -- really any value\\ (in Python anyway)
  \item[attribute] something that belongs to an object (or class)
    -- generally thought of as a variable, or single object, as apposed to a ...
  \item[method] a function that belongs to a class
\end{description}

\end{frame} 


\section{Python Classes}

% ---------------------------------------------
\begin{frame}[fragile]{Python Classes}

{\Large The \verb|class| statement}

\vfill
{\large \verb|class| creates a new type object:}

\begin{verbatim}
In [4]: class C(object):
    pass
   ...: 
In [5]: type(C)
Out[5]: type
\end{verbatim}

{\large It is created when the statement is run -- much like \verb|def|}

\vfill
(note on``new style'' classes)

\end{frame} 

% ---------------------------------------------
\begin{frame}[fragile]{Python Classes}

{\Large Note about the book (TP):}

Chapters 15 and 16 use a style that generally isn't recommended:

\begin{verbatim}
In [6]: class Point(object):
   ...:     pass
In [7]: p = Point()
In [8]: p.x = 4
In [9]: p.y = 2
\end{verbatim}

Python is Dynamic -- you can do this, but you generally want more structure,
defaults, etc. 

\vfill
(it used to be a quick and dirty "struct" -- but use a named tuple now)
\end{frame} 

% ---------------------------------------------
\begin{frame}[fragile]{Python Classes}

{\Large About the simplest class:}

\begin{verbatim}
>>> class Point:
...     x = 1
...     y = 2
>>> Point
<class __main__.Point at 0x2bf928>
>>> p
<__main__.Point instance at 0x2de918>
>>> Point.x
1
>>> p = Point()
>>> p.x
1
\end{verbatim}
\end{frame} 

% ---------------------------------------------
\begin{frame}[fragile]{Python Classes}

{\Large Basic Structure of a real class}

\begin{verbatim}
class Point(object):
# everything defined in here is in the class namespace
    def __init__(self, x, y):
        self.x = x
        self.y = y

## create an instance of that class        
p = Point(3,4)

## access the attributes
print "p.x is:", p.x
print "p.y is:", p.y
\end{verbatim}

see: \verb|simple_class| in \verb|code| dir
\end{frame} 


% ---------------------------------------------
\begin{frame}[fragile]{Python Classes}

{\LARGE The Initializer}

\vfill
{\Large The \verb|__init__| special method is called when a new instance of a class is created.}

\vfill
{\Large You can use it to do any set-up you need}

\vfill
\begin{verbatim}
class Point(object):
    def __init__(self, x, y):
        self.x = x
        self.y = y
\end{verbatim}
\vfill
{\Large It gets the arguments passed to the class constructor}
\end{frame} 

% ---------------------------------------------
\begin{frame}[fragile]{Python Classes}

{\LARGE \verb|self|}

\vfill
{\Large The instance of the class is passed as the first parameter for every method.}

\vfill
{\Large ``\verb|self|'' is only a convention -- but you DO want to use it.}

\vfill
\begin{verbatim}
class Point(object):
    def a_function(self, x, y):
...
\end{verbatim}
\vfill
{\Large Does this look familiar from C-style procedural programming?}
\end{frame} 


% ---------------------------------------------
\begin{frame}[fragile]{Python Classes}

\begin{verbatim}
class Point(object):
    def __init__(self, x, y):
        self.x = x
        self.y = y
\end{verbatim}

\vfill
{\Large Anything assigned to a \verb|self.| attribute is kept in the instance
name space}

\vfill
{\Large That's where all the instance-specific data is.}

\vfill
\end{frame} 

% ---------------------------------------------
\begin{frame}[fragile]{Python Classes}

\begin{verbatim}
class Point(object):
    size = 4
    color= "red"
    def __init__(self, x, y):
        self.x = x
        self.y = y
\end{verbatim}

\vfill
{\Large Anything assigned in the class scope is a class attribute -- every
instance of the class shares the same one.}
\vfill
\end{frame} 

% ---------------------------------------------
\begin{frame}[fragile]{Python Classes}

\begin{verbatim}
class Point(object):
    size = 4
    color= "red"
...
    def get_color():
        return self.color

>>> p3.get_color()
 'red'
\end{verbatim}

\vfill
{\Large class attributes are accessed with \verb|self| also..} 
\vfill
\end{frame} 


% ---------------------------------------------
\begin{frame}[fragile]{Python Classes}

{\Large Typical methods}
\begin{verbatim}
class Circle(object):
    color = "red"
    def __init__(self, diameter):
        self.diameter = diameter

    def grow(self, factor=2):
        self.diameter = self.diameter * factor
\end{verbatim}

\vfill
{\Large methods take some parameters, manipulate the attributes in \verb|self|} 

\end{frame} 

% ---------------------------------------------
\begin{frame}[fragile]{Python Classes}

{\Large Gotcha!}
\begin{verbatim}
...
    def grow(self, factor=2):
        self.diameter = self.diameter * factor
...

In [205]: C = Circle(5)
In [206]: C.grow(2,3)

TypeError: grow() takes at most 2 arguments (3 given)

\end{verbatim}

\vfill
{\LARGE Huh???? I only gave 2} 
\end{frame} 


%%-------------------------------
\begin{frame}[fragile]{LAB}

\vfill
{\Large We had such a good time last class -- we'll do something similar}

\vfill
{\Large The goal is to build a set of classes that render an html page:
\verb|sample_html.html|
}

\vfill
{\Large We'll start with a single class, then add some sub-classes to specialize the behavior}

\vfill
More details in \verb+week-06/LAB_instuctions.txt+
\end{frame}

%%-------------------------------
\begin{frame}[fragile]{LAB}

\vfill
{\Large Step 1:}

\begin{itemize}
  \item Create an "Element" class for rendering an html element (xml element). 
  \item It should have class attributes for the tag name  and the
  indentation
  \item the constructor signature should look like:
    \verb|Element(content=None)| where content is a string
  \item It should have an "append" method that can add another string to the content
  \item It should have a \verb|render(file_out, ind = "")| method that renders the tag
     and the strings in the content.

     \verb|file_out| could be any file-like object.

     \verb|ind| is a string with enough spaces to indent properly.
\end{itemize}

\end{frame}

%-------------------------------
\begin{frame}{Lightning Talk}

{\centering

\vfill
{\LARGE Lightning Talk:  }

\vfill
{\Huge Brett}

\vfill
}
\end{frame}

\section{Subclassing/Inheritance}

% ---------------------------------------------
\begin{frame}[fragile]{Inheritance}

In object-oriented programming (OOP), inheritance is a way to reuse code of
existing objects, or to establish a subtype from an existing object.

\vfill
...

\vfill
objects are defined by classes, classes can inherit attributes and behavior
from pre-existing classes called base classes, or super classes.

\vfill
The resulting classes are known as derived classes or subclasses.

\vfill
(\url{http://en.wikipedia.org/wiki/Inheritance_%28object-oriented_programming%29})
\end{frame} 

% ---------------------------------------------
\begin{frame}[fragile]{Subclassing}

A subclass ``inherits'' all the attributes (methods, etc) of the parent class.

\vfill
You can then change (``override'') some or all of the attributes to change the behavior.

\vfill
The simplest subclass in Python:

\begin{verbatim}
class A_Subclass(The_SuperClass):
    pass
\end{verbatim}

\vfill
\verb|A_subclass| now has exactly the same behavior as \verb|The_SuperClass|

\end{frame} 

% ---------------------------------------------
\begin{frame}[fragile]{Overriding attributes}

{\Large Overriding is as simple as creating a new attribute with the same name:}

\vfill
\begin{verbatim}
class Circle(object):
    color = "red"
...
class NewCircle(Circle):
    color = "blue"
>>> nc = NewCircle
>>> print nc.color
blue
\end{verbatim}

\vfill
all the \verb|self| instances will have the new attribute
\end{frame} 

% ---------------------------------------------
\begin{frame}[fragile]{Overriding methods}

{\Large Same thing, but with methods}

\vfill
\begin{verbatim}
class Circle(object):
...
    def grow(self, factor=2):
        """grows the circle's diameter by factor"""
        self.diameter = self.diameter * factor
...
class NewCircle(Circle):
...
    def grow(self, factor=2):
        """grows the area by factor..."""
        self.diameter = self.diameter * math.sqrt(2)
\end{verbatim}
all the instances will have the new method
\end{frame} 

\begin{frame}

{\Large
``Here's a program design suggestion: whenever you override a method, the
interface of the new method should be the same as the old.  It should take
the same parameters, return the same type, and obey the same preconditions
and postconditions.  If you obey this rule, you will find that any function
designed to work with an instance of a superclass, like a Deck, will also work
with instances of subclasses like a Hand or PokerHand.  If you violate this
rule, your code will collapse like (sorry) a house of cards.''
}
\vfill
\hfill ThinkPython 18.10
\end{frame}

%%-------------------------------
\begin{frame}[fragile]{LAB}

\vfill
{\Large Step 2:}

\begin{itemize}
  \item  Create a couple subclasses of \verb|Element|, for a \verb|<body>| tag
         and \verb|<p>| tag. Simply override the \verb|tag| class attribute.
  \item Extend the \verb|Element.render()| method so that it can render other
        elements inside the tag in addition to strings. Simple recursion should
        do it. i.e. it can call the \verb|render()| method of the elements it
        contains.
  \item Deal with the content items that could be either simple strings or
        \verb|Element|s with \verb|render| methods...there are a few ways to handle that...
\end{itemize}

\end{frame}


%%-------------------------------
\begin{frame}[fragile]{LAB}

\vfill
{\Large Step 3:}

\begin{itemize}
  \item Create a \verb|<head>| element -- simple subclass.
  \item Create a \verb|OneLineTag| subclass of Element:
        It should override the render method, to render everything on one line --
        for the simple tags, like:
    
        \verb|<title> PythonClass - Class 6 example </title>|
  \item Create a Title subclass of \verb|OneLineTag| class for the title.
  
  \item You should now be able to render an html doc with a head element, with
       a \verb|title| element in that, and a body element with some \verb|<P>|
       elements and some text.
\end{itemize}

\end{frame}


%-------------------------------
\begin{frame}{Lightning Talk}

{\centering

\vfill
{\LARGE Lightning Talk:  }

\vfill
{\Huge Matt}

\vfill
}
\end{frame}

% ---------------------------------------------
\begin{frame}[fragile]{Overriding \_\_init\_\_}

{\Large \verb|__init__|common method to override}
\vfill
{\large You often need to call the super class \verb|__init__| as well}
\vfill
\begin{verbatim}
class Circle(object):
    color = "red"
    def __init__(self, diameter):
        self.diameter = diameter
...
class CircleR(Circle):
    def __init__(self. radius):
        diameter = radius*2
        Circle.__init__(self, diameter)
\end{verbatim}
\vfill
exception to: "don't change the method signature" rule.
\end{frame} 

% ---------------------------------------------
\begin{frame}[fragile]{Overriding other methods}

{\large You can also call the superclasses other methods:}
\vfill
\begin{verbatim}
class Circle(object):
...
    def get_area(self, diameter):
        return math.pi * diameter / 2.0

class CircleR2(Circle):
...
    def get_area(self):
        return Circle.get_area(self, self.radius*2)
\end{verbatim}

\vfill
There is nothing special about \verb|__init__| except that it gets called automatically.
\end{frame} 

\begin{frame}[fragile]{When to Subclass}

\vfill
{\Large ``Is a'' relationship: Subclass/inheritance}

\vfill
{\Large ``Has a'' relationship: Composition}
\end{frame}

\begin{frame}[fragile]{When to Subclass}

{\Large ``Is a'' vs ``Has a'' }

\vfill
You may have a class that needs to accumulate an arbitrary number of objects.

\vfill
A list can do that -- do should you subclass list?

\vfill
Ask yourself:\\

\vfill
-- IS you class a list (with some extra functionality)?\\
\hspace{0.4in}or\\
-- Does you class HAVE a list?\\

\vfill
You only want to subclass list if your class could be used anywhere is list is.
\end{frame}

% ---------------------------------------------
\begin{frame}[fragile]{Attribute resolution order}

{\Large When you access an attribute:

\vfill
\hspace{0.2in}\verb|An_Instance.something|}

\vfill
{\Large Python looks for it in this order:}

\vfill
\begin{enumerate}
  \item Is it an instance attribute ?
  \item Is it a class attribute ?
  \item Is it a superclass attribute ?
  \item Is it a super-superclass attribute ?
  \item ...
\end{enumerate}

\vfill
It can get more complicated...\\
\url{http://www.python.org/getit/releases/2.3/mro/}
\end{frame} 

% ---------------------------------------------
\begin{frame}[fragile]{What are Python classes, really?}

{\Large Putting aside the OO theory...}

\vfill
{\Large Python classes are:}

\begin{itemize}
  \item Namespaces
  \begin{itemize}
    \item One for the class object
    \item One for each instance
  \end{itemize}
  \item Attribute resolution order
  \item Auto tacking-on of \verb|self|
\end{itemize}

\vfill
{\Large That's about it -- really!}

\end{frame} 

% ---------------------------------------------
\begin{frame}[fragile]{Type-Based dispatch}

{\Large From Think Python:}

\begin{verbatim}
  if isinstance(other, A_Class):
      Do_something_with_other
  else:
      Do_something_else
\end{verbatim}

\vfill
{\Large Usually better to use ``duck typing'' (polymorphism)}

\vfill
{\Large But when it's called for:}
\begin{itemize}
    \item \verb|isinstance()|
    \item \verb|issubclass()|
\end{itemize}

\vfill
GvR: ``Five Minute Multi- methods in Python'':\\
{\small \url{http://www.artima.com/weblogs/viewpost.jsp?thread=101605} }

\end{frame} 

%-------------------------------
\begin{frame}[fragile]{LAB}

{\Large We're going to the rest: steps 4 - 8}

\vfill
{\Large Step 4:}

\begin{itemize}
  \item Extend the Element class to accept a set of attributes as keywords to the
  constructor, i.e.:
  \begin{verbatim}
Element("some text content",
        id="TheList",
        style="line-height:200\%")
  \end{verbatim}
    ( remember \verb|**kwargs| ? )
  \item The render method will need to be extended to render the attributes properly.
\end{itemize}

\vfill
You can now render some \verb|<p>| tags (and others) with attributes  
\end{frame}

%-------------------------------
\begin{frame}[fragile]{LAB}

{\Large Step 5:}

\begin{itemize}
   \item Create a "SelfClosingTag" subclass of Element, to render tags like:
   
   \verb|<meta charset="UTF-8" />|
   
   \item You will need to override the render method to render just the one tag and
   attributes.
   
   \item create a couple subclasses of SelfClosingTag for \verb|<meta>| and \verb|<hr>|
   (horizontal rule) -- and \verb|<br />| or ??? if you like
   \end{itemize}

\vfill
You can now render an html page with a proper \verb|<head>| (\verb|<meta />| and \verb|<title>| elements)
\end{frame}

\begin{frame}[fragile]{LAB}

{\Large Step 6:}

\begin{itemize}
   \item  Create an A class for an anchor (link) element. Its constructor should
          look like: \verb|A(self, link, content)| -- where link is the link,
          and content is what you see. It can be called like so:

   
   \verb|A("http://google.com", "link")|

  \item You should be able to subclass from \verb|Element|, and only override
        the \verb|__init__|\\
        -- Calling the \verb|Element __init__| from the  \verb|A __init__|
\end{itemize}

\vfill
    You can now add a link to your web page.
\end{frame}

\begin{frame}[fragile]{LAB}

{\Large Step 7:}

\begin{itemize}
   \item Create \verb|Ul| class for an unordered list (really simple subclass of Element)
   
   \item Create \verb|Li| class for an element in a list (also really simple)
   
   \item add a list to your web page.
   
   \item Create a Header class -- this one should take an integer argument for the
   header level. i.e \verb|<h1>, <h2>, <h3>|, called like:
   
   \item \verb|H(2, "The text of the header")| for an \verb|<h2>| header
   
   \item It can subclass from \verb|OneLineTag| -- overriding the \verb|__init__|, then calling
       the superclass \verb|__init__|
\end{itemize}

\end{frame}

\begin{frame}[fragile]{LAB}

{\Large Step 8:}

\begin{itemize}
   \item Update the Html element class to render the "\verb|<!DOCTYPE html>|" tag at the
   head of the page, before the \verb|html| element.
   
   \item You can do this by subclassing \verb|Element|, overriding \verb|render()|, but then
   calling the \verb|Element render()| from the new \verb|render()|.
\end{itemize}

\vfill
You now have a pretty full-featured html renderer -- play with it, add some
new tags, etc....Maybe integrate it into your http server...
\end{frame}



% ---------------------------------------------
\begin{frame}[fragile]{multiple inheritance}

{\Large Multiple inheritance: Pulling from more than one class}

\vfill
\begin{verbatim}
class Combined(Super1, Super2, Super3):
    def __init__(self, something, something else):
        Super1.__init__(self, ......)        
        Super2.__init__(self, ......)        
        Super3.__init__(self, ......)        
\end{verbatim}
(calls to the super class \verb|__init__| are optional -- usage dependent)

\vfill
{\Large Attribute resolution -- right to left}

\vfill
{\Large Why would you want to do this?}

\end{frame} 

% ---------------------------------------------
\begin{frame}[fragile]{mix-ins}

\vfill
{\Large Hierarchies are not always simple}
\vfill
\begin{itemize}
  \item Animal
  \begin{itemize}
    \item Mammal
    \begin{itemize}
      \item GiveBirth()
    \end{itemize}
    \item Bird
    \begin{itemize}
      \item LayEggs()
    \end{itemize}
  \end{itemize}
\end{itemize}
\vfill
{\Large Where do you put a Platapus or an Armadillo?}

\vfill
{\Large Real World Example: \verb|FloatCanvas|}
\end{frame} 

%%-------------------------------
%\begin{frame}{LAB}
%
%\begin{itemize}
%  \item
%\end{itemize}
%
%\end{frame}
%

\end{document}

 
