\documentclass{beamer}
%\usepackage[latin1]{inputenc}
\usetheme{Warsaw}
\title[Intro to Python: Week 7]{Introduction  to Python\\ More OO -- Decorators and Generators}
\author{Christopher Barker}
\institute{UW Continuing Education / Isilon}
\date{August 08, 2012}

\usepackage{listings}
\usepackage{hyperref}

\begin{document}

% ---------------------------------------------
\begin{frame}
  \titlepage
\end{frame}

% ---------------------------------------------
\begin{frame}
\frametitle{Table of Contents}
%\tableofcontents[currentsection]
  \tableofcontents
\end{frame}

%-------------------------------
\begin{frame}{Lightning Talks}

{\centering

\vfill
{\LARGE Lightning Talks today:  }

\vfill
{\Huge Brett and Jeffery}

\vfill
}
\end{frame}

\section{Review/Questions}

% ---------------------------------------------
\begin{frame}[fragile]{Review of Previous Class}

\begin{itemize}
  \item Really Quick OO overview
  \item Built an html generator, using: 
  \begin{itemize}
    \item A base class with a couple methods
    \item Subclasses overriding class attributes
    \item Subclasses overriding a method
    \item Subclasses overriding the \verb|__init__|
  \end{itemize}
\end{itemize}

\end{frame}


%% ---------------------------------------------
\begin{frame}{Homework review}

\vfill
{\LARGE Homework notes }

\vfill
{\Large Did anyone make classes for a real project? }

\vfill
{\Large How many of you finished the html building code from last class? }
\vfill

\end{frame}

\section{More OO}

% ---------------------------------------------
\begin{frame}[fragile]{multiple inheritance}

{\Large Multiple inheritance:\\
\hspace{0.2in} Pulling from more than one class}

\vfill
\begin{verbatim}
class Combined(Super1, Super2, Super3):
    def __init__(self, something, something else):
        Super1.__init__(self, ......)        
        Super2.__init__(self, ......)        
        Super3.__init__(self, ......)        
\end{verbatim}
(calls to the super class \verb|__init__| are optional -- case dependent)

\end{frame} 

% ---------------------------------------------
\begin{frame}[fragile]{multiple inheritance}

\vfill
{\Large Attribute resolution -- left to right}

\begin{enumerate}
  \item Is it an instance attribute ?
  \item Is it a class attribute ?
  \item Is it a superclass attribute ?
  \begin{enumerate}
     \item is the it an attribute of the left-most superclass?
     \item is the it an attribute of the next superclass?
     \item ....
  \end{enumerate}
  \item Is it a super-superclass attribute ?
  \item ...also left to right...
\end{enumerate}

\end{frame} 

% ---------------------------------------------
\begin{frame}[fragile]{mix-ins}

{\Large Why would you want to do this?}

\vfill
{\Large Hierarchies are not always simple:}
\vfill
\begin{itemize}
  \item Animal
  \begin{itemize}
    \item Mammal
    \begin{itemize}
      \item GiveBirth()
    \end{itemize}
    \item Bird
    \begin{itemize}
      \item LayEggs()
    \end{itemize}
  \end{itemize}
\end{itemize}
\vfill
{\Large Where do you put a Platypus or an Armadillo?}

\vfill
{\Large Real World Example: \verb|FloatCanvas|}
\end{frame} 

\begin{frame}[fragile]{LAB}

{\Large Last class: Step 6:}

\begin{itemize}
   \item  Create an \verb|A| class for an anchor (link) element. Its constructor should
          look like: \verb|A(self, link, content)| -- where link is the link,
          and content is what you see. It can be called like so:

   \verb|A("http://google.com", "link")|

  \item You should be able to subclass from \verb|Element|, and only override
        the \verb|__init__|\\
        -- Calling the \verb|Element __init__| from the  \verb|A __init__|
\end{itemize}

\vfill
    You can now add a link to your web page.
\end{frame}


\begin{frame}[fragile]{LAB}

{\Large Last class: Step 7:}

\begin{itemize}
   \item Create \verb|Ul| class for an unordered list (really simple subclass of Element)
   
   \item Create \verb|Li| class for an element in a list (also really simple)
   
   \item add a list to your web page.
   
   \item Create a Header class -- this one should take an integer argument for the
   header level. i.e \verb|<h1>, <h2>, <h3>|, called like:
   
   \item \verb|H(2, "The text of the header")| for an \verb|<h2>| header
   
   \item It can subclass from \verb|OneLineTag| -- overriding the \verb|__init__|, then calling
       the superclass \verb|__init__|
\end{itemize}

\end{frame}

\begin{frame}[fragile]{LAB}

{\Large Last class Step 8:}

\begin{itemize}
   \item Update the Html element class to render the "\verb|<!DOCTYPE html>|" tag at the
   head of the page, before the \verb|html| element.
   
   \item You can do this by subclassing \verb|Element|, overriding \verb|render()|, but then
   calling \verb|Element.render()| from \verb|Html.render()|.

   \item Create a subclass of \verb|SelfClosingTag| for \verb|<meta charset="UTF-8" />|
   and add the meta element to the beginning of the head element to give your document
   an encoding.
   
   \item The doctype and encoding are HTML 5 and you can check this at:
          \url{validator.w3.org.}

\end{itemize}

\vfill
You now have a pretty full-featured html renderer
\end{frame}

%-------------------------------
\begin{frame}{Lightning Talk}

{\centering

\vfill
{\LARGE Lightning Talk:  }

\vfill
{\Huge Brett}

\vfill
}
\end{frame}

\section{Properties}

% ---------------------------------------------
\begin{frame}[fragile]{Accessing Attributes}

{\Large One of the strengths of Python is lack of clutter}

\vfill
{\Large Simple attributes:}

\begin{verbatim}
In [5]: class C(object):
        def __init__(self):
                self.x = 5
In [6]: c = C()
In [7]: c.x
Out[7]: 5
In [8]: c.x = 8
\end{verbatim}

\end{frame} 

% ---------------------------------------------
\begin{frame}[fragile]{Getter and Setters?}

{\Large What if you need to add behavior later?}

\begin{itemize}
  \item do some calculation
  \item check data validity
  \item keep things in sync
\end{itemize}

\end{frame}


\begin{frame}[fragile]{Getter and Setters?}

\begin{verbatim}
class C(object):
    def get_x(self):
        return self.x
    def set_x(self, x):
        self.x = x
>>> c = C()
>>> c.get_x()
>>> 5
>>> c.set_x(8)
>>> c.get_x()
>>> 8
\end{verbatim}
{\Large Ugly and verbose -- Java?}

\url{http://dirtsimple.org/2004/12/python-is-not-java.html}

\end{frame} 

\begin{frame}[fragile]{properties}

{ \Large When (and if) you need them: }

\begin{verbatim}
class C(object):
    def getx(self):
        return self._x
    def setx(self, value):
        self._x = value
    def delx(self):
        del self._x
    x = property(getx, setx, delx, "docstring")
\end{verbatim}
{\Large Interface is still like simple attribute access}
(\verb|properties_sample.py| )
\end{frame} 


\begin{frame}[fragile]{properties}

{ \Large When (and if) you need them: }

\begin{verbatim}
class C(object):
    def getx(self):
        return self._x
    def setx(self, value):
        self._x = value
    def delx(self):
        del self._x
    x = property(getx, setx, delx, "docstring")
\end{verbatim}
{\Large Interface is still like simple attribute access}
(\verb|properties_sample.py| )
\end{frame} 

\begin{frame}[fragile]{staticmethod}

{ \Large A method that doesn't get self! }

\begin{verbatim}
class C(object):
    def add(a, b):
        return a + b
    add = staticmethod(add)
>>> C.add(3,4)
7
>>> c = C()
>>> c.add(2, 2)
4
\end{verbatim}
{\Large When you don't need self -- can be used from either an instance or the class itself}

\vfill
see: \verb|static_method.py|
\end{frame} 

\begin{frame}[fragile]{classmethod}

{ \Large Method gets the class object, rather than an instance the first argument}

\begin{verbatim}
class C(object):
    def __init__(self, x, y):
        self.x = x
        self.y = y
    def a_class_method(klass, y):
        print "in a_class_method", klass
        return klass( y, y**2 )
    a_class_method = classmethod(a_class_method)
\end{verbatim}
{\Large When you need the class object rather than an instance -- plays well with subclassing}
\vfill
see: \verb|class_method.py|
\end{frame} 

\begin{frame}[fragile]{dict.fromkeys()}

{ \Large \verb|classmethod| often used for alternate constructors:}

\begin{verbatim}
>>> d = dict([1,2,3])
Traceback (most recent call last):
  File "<stdin>", line 1, in <module>
TypeError: cannot convert dictionary update
sequence element #0 to a sequence
>>> d = dict.fromkeys([1,2,3])
>>> d
{1: None, 2: None, 3: None}
\end{verbatim}

\end{frame} 

\begin{frame}[fragile]{dict.fromkeys()}

\begin{verbatim}
class Dict: ...
    def fromkeys(klass, iterable, value=None):
        "Emulate dict_fromkeys() in dictobject.c"
        d = klass()
        for key in iterable:
            d[key] = value
        return d
    fromkeys = classmethod(fromkeys)
\end{verbatim}

\vfill
{\Large See also datetime.datetime.now(), etc....}

\vfill
For a low-level look:\\
\url{http://docs.python.org/howto/descriptor.html}

\end{frame} 

\begin{frame}[fragile]{super}

{\Large getting the superclass:}
\begin{verbatim}
class SafeVehicle(Vehicle):
    """
    Safe Vehicle subclass of Vehicle base class...
    """
    def __init__(self, position=0, velocity=0, icon='S'):
        Vehicle.__init__(self, position, velocity, icon)
\end{verbatim}

{\Large
\vfill
not DRY
\vfill
also, what if we had a bunch of references to superclass?
}
\end{frame} 

\begin{frame}[fragile]{super}

{\Large getting the superclass:}
\begin{verbatim}
class SafeVehicle(Vehicle):
    """
    Safe Vehicle subclass of Vehicle base class
    """
    def __init__(self, position=0, velocity=0, icon='S'):
        super(SafeVehicle, self).__init__(position, velocity, icon)
\end{verbatim}

\vfill
{\Large ``super() considered super!'' by Raymond Hettinger }
\vfill
\url{http://rhettinger.wordpress.com/2011/05/26/super-considered-super/}

\end{frame} 

\section{Special Attributes}

\begin{frame}[fragile]{special methods}

{\Large Python's Duck typing:}

\vfill
{\Large Defining special (or magic) methods in your classes is how you make
your class act like standard classes}

\end{frame} 

\begin{frame}[fragile]{special methods}

{\Large We've seen at least one:}

\begin{verbatim}
__init__
\end{verbatim}

\vfill
{\large it's all in the double underscores...}

\vfill
{\large Pronounced ``dunder'' (or ``under-under'') }

\vfill
try: \verb|dir(2)| or \verb|dir(list)|
\end{frame} 

\begin{frame}[fragile]{special methods}

{\Large Emulating Numeric types}

\begin{verbatim}
object.__add__(self, other)
object.__sub__(self, other)
object.__mul__(self, other)
object.__floordiv__(self, other)
object.__mod__(self, other)
object.__divmod__(self, other)
object.__pow__(self, other[, modulo])
object.__lshift__(self, other)
object.__rshift__(self, other)
object.__and__(self, other)
object.__xor__(self, other)
object.__or__(self, other)¶
\end{verbatim}

\end{frame} 

\begin{frame}[fragile]{special methods}

{\Large Emulating container types:}

\begin{verbatim}
object.__len__(self)
object.__getitem__(self, key)
object.__setitem__(self, key, value)
object.__delitem__(self, key)
object.__iter__(self)
object.__reversed__(self)
object.__contains__(self, item)
object.__getslice__(self, i, j)
object.__setslice__(self, i, j, sequence)
object.__delslice__(self, i, j)
\end{verbatim}

\end{frame} 


\begin{frame}[fragile]{special methods}

{\Large You get the idea...}

\vfill
{\Large You only need to define the ones that are going to get used}

\vfill
{\Large but you probably want to define at least these:}

\vfill
\verb|object.__str__|: Called by the str() built-in function and by the print statement to compute the “informal” string representation of an object.

\vfill
\verb|object.__repr__|: Called by the repr() built-in function and by string conversions (reverse quotes) to compute the “official” string representation of an object.

\end{frame} 

\begin{frame}[fragile]{special methods}

\vfill
{\Large When you want your class to act like a "standard" calss in some way:}

\vfill
{\Large Look up the magic methods you need and define them}

\vfill
\url{http://docs.python.org/reference/datamodel.html#special-method-names}

\vfill
\url{http://www.rafekettler.com/magicmethods.html}
\end{frame} 



\section{Iterators / Generators}

% ---------------------------------------------
\begin{frame}[fragile]{Iterators}

{\Large Iterators are one of the main reasons Python code is so readable:}

\begin{verbatim}
for x in just_about_anything:
    do_stuff(x)
\end{verbatim}

{\Large you can loop through anything that satisfies the iterator protocol}

\vfill
\url{http://docs.python.org/library/stdtypes.html#iterator-types}
\end{frame} 

\begin{frame}[fragile]{Iterator Protocol}

{\Large An iterator must have the following methods:}

\begin{verbatim}
iterator.__iter__()
\end{verbatim}

Return the iterator object itself. This is required to allow both containers
and iterators to be used with the for and in statements.

\begin{verbatim}
iterator.next()
\end{verbatim}

Return the next item from the container. If there are no further items,
raise the StopIteration exception.

\end{frame} 


\begin{frame}[fragile]{Example Iterator}

\begin{verbatim}
class IterateMe_1(object):
    def __init__(self, stop=5):
        self.current = 0
        self.stop = 5
    def __iter__(self):
        return self
    def next(self):
        if self.current < self.stop:
            self.current += 1
            return self.current
        else:
            raise StopIteration
\end{verbatim}

{\Large This is a simple version of \verb|xrange()|}

\end{frame} 

%------------------------------
\begin{frame}[fragile]{itertools}

{\Large \verb|itertools| is a collection of utilities that make it easy to
build an iterator that iterates over sequences in various common ways}

\begin{verbatim}

\end{verbatim}

\url{http://docs.python.org/library/itertools.html}

\end{frame}

%%-------------------------------
\begin{frame}[fragile]{LAB}

\begin{itemize}
  \item  Extend (\verb|iterator_1.py|) to be more like \verb|xrange()| --
         add three input parameters: \verb|iterator_2(start, stop, step=1)|
  \item  See what happens if you break out in the middle of the loop:
\begin{verbatim}
    it = IterateMe_2(2, 20, 2)
    for i in it:
        if i > 10:  break
        print i
\end{verbatim}
And then pick up again:
\begin{verbatim}
    for i in it:
        print i
\end{verbatim}
  \item  Does \verb|xrange()| behave the same?\\
          -- make yours match \verb|xrange()|.
\end{itemize}
\end{frame}

%%-------------------------------
\begin{frame}[fragile]{generators}

\Large{Generators give you the iterator immediately
no access to the underlying data ... if it even exists}

\vfill
\Large{ Conceptually:}

iterators are about various ways to loop over data,

generators generate the data on the fly

\vfill
\Large{ Practically:}

You can use either either way (and a generator is one type of iterator)

Generators do some of the book-keeping for you.

\end{frame}

%%-------------------------------
\begin{frame}[fragile]{yield}

\Large{\verb|yield| is a way to make a quickie generator with a function:}

\begin{verbatim}
def a_generator_function(params):
    some_stuff
    yield(something)
\end{verbatim}

\vfill
\Large{ Generator functions "yield" a value, rather than returning it }

\vfill
\Large{ State is preserved in between yields }

\end{frame}

%%-------------------------------
\begin{frame}[fragile]{yield}

\Large{A function with \verb|yield| in it is a ``factory'' for a generator}

\Large{Each time you call it, you get a new generator:}
 
\begin{verbatim}
gen_a = a_generator()
gen_b = a_generator()
\end{verbatim}

\vfill
\Large{ Each instance keeps its own state. }

\vfill
\Large{ Really just a shorthand for an iterator class that does the book keeping for you.}

\end{frame}

%%-------------------------------
\begin{frame}[fragile]{yield}

\Large{An example: like \verb|xrange()|}

\begin{verbatim}
def y_xrange(start, stop, step=1):
    i = start
    while i < stop:
        yield i
        i += step
\end{verbatim}

\vfill
{\Large Real World Example: \verb|FloatCanvas|}
\end{frame}

%%-------------------------------
\begin{frame}[fragile]{yield}

{\Large Note:}

\begin{verbatim}
In [164]: gen = y_xrange(2,6)

In [165]: type(gen)
Out[165]: generator

In [166]: dir(gen)
Out[166]: 
...
 '__iter__',
...
 'next',
\end{verbatim}
{\Large So the generator {\bf is} an iterator}
\end{frame}

%%-------------------------------
\begin{frame}[fragile]{yield}

{\Large A generator function can also be a method in a class}

\vfill
{\Large More about iterators and generators:}

\url{http://www.learningpython.com/2009/02/23/iterators-iterables-and-generators-oh-my/}

\vfill
\verb|yield_example.py|
\end{frame}

%%-------------------------------
\begin{frame}[fragile]{generator comprehension}

{\Large another way to make a generator:}

\begin{verbatim}
>>> [x * 2 for x in [1, 2, 3]]
[2, 4, 6]
>>> (x * 2 for x in [1, 2, 3])
<generator object <genexpr> at 0x10911bf50>
>>> for n in (x * 2 for x in [1, 2, 3]):
...   print n
... 2 4 6
\end{verbatim}

\vfill
More interesting if [1, 2, 3] is also a generator

\end{frame}


%%-------------------------------
\begin{frame}[fragile]{LAB}

{\Large generator lab}

\end{frame}

%-------------------------------
\begin{frame}{Lightning Talk}

{\centering

\vfill
{\LARGE Lightning Talk:  }

\vfill
{\Huge Jeffery}

\vfill
}
\end{frame}

%-------------------------------
\begin{frame}[fragile]{Homework}

\vfill
{\Large Pickup from last week -- For a portion of the system you're responsible for testing:}
\begin{itemize}
  \item Make a few classes to represent that portion. (No more than three or
        four classes)
  \item Try to make use or something from today:
  \begin{itemize}
    \item mix-ins
    \item properties
    \item Magic methods
    \item iterator or generator
  \end{itemize}
\end{itemize}

\vfill
\end{frame}


\end{document}

 
