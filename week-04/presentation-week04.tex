\documentclass{beamer}
%\usepackage[latin1]{inputenc}
\usetheme{Warsaw}
\title[Intro to Python: Week 4]{Introduction  to Python}
\author{Christopher Barker}
\institute{UW Continuing Education / Isilon}
\date{July 18, 2012}

\usepackage{listings}
\usepackage{hyperref}

\begin{document}

% ---------------------------------------------
\begin{frame}
  \titlepage
\end{frame}

% ---------------------------------------------
\begin{frame}
\frametitle{Table of Contents}
%\tableofcontents[currentsection]
  \tableofcontents
\end{frame}


\section{Review/Questions}

% ---------------------------------------------
\begin{frame}{Review of Previous Class}

\begin{itemize}
  \item ...
\end{itemize}

\end{frame}


% header
% ---------------------------------------------
\begin{frame}{Homework review}

  {\Large Homework notes }

\end{frame}

\section{More on function calling}

% ---------------------------------------------
\begin{frame}[fragile]{Default Parameters}

 {\Large Sometimes you don't need the user to specify everything every time}

\begin{verbatim}
In [142]: def fun(x,y,z=5):
   .....:         print x,y,z

In [143]: fun(1,2)
1 2 5

In [144]: fun(1,2,3)
1 2 3
\end{verbatim}

\end{frame} 

% ---------------------------------------------
\begin{frame}[fragile]{Keyword arguments}

 {\Large You can specify only what you need -- any order}

\begin{verbatim}
In [151]: def fun(x,y=0,z=0):
        print x,y,z
   .....:     

In [152]: fun(1,2,3)
1 2 3

In [153]: fun(1, z=3)
1 0 3

In [154]: fun(1, z=3, y=2)
1 2 3
\end{verbatim}

\end{frame} 

% ---------------------------------------------
\begin{frame}[fragile]{Keyword arguments}

 {\Large You can specify only what you need -- any order}

\begin{verbatim}
In [151]: def fun(x,y=0,z=0):
        print x,y,z
   .....:     

In [152]: fun(1,2,3)
1 2 3

In [153]: fun(1, z=3)
1 0 3

In [154]: fun(1, z=3, y=2)
1 2 3
\end{verbatim}

\end{frame} 

% ---------------------------------------------
\begin{frame}[fragile]{Keyword arguments}

 {\Large A Common Idiom}

\vfill
\begin{verbatim}
def fun(x,y=None):
    if y is None:
        do_something_different

    go_on_here
\end{verbatim}
\vfill

\end{frame} 

% ---------------------------------------------
\begin{frame}[fragile]{Keyword arguments}

 {\Large Can set defaults to variables}

\begin{verbatim}
In [156]: y = 4

In [157]: def fun(x=y):
    print "x is:", x
   .....:     

In [158]: fun()
x is: 4

In [159]: y = 6

In [160]: fun()
x is: 4
\end{verbatim}

\end{frame} 

% ---------------------------------------------
\begin{frame}[fragile]{Keyword arguments}

{\Large Defaults are evaluated when the function is defined}

\begin{verbatim}
In [156]: y = 4

In [157]: def fun(x=y):
    print "x is:", x
   .....:     

In [158]: fun()
x is: 4

In [159]: y = 6

In [160]: fun()
x is: 4
\end{verbatim}

\end{frame} 

% ---------------------------------------------
\begin{frame}[fragile]{Keyword arguments}

\begin{verbatim}
In [161]: l = []
In [162]: for i in range(3):
   .....:     def fun(x=i):
   .....:         print x
   .....:     l.append(fun)
In [163]: l
Out[163]: [<function __main__.fun>, <function __main__.fun>, <function __main__.fun>]
In [164]: l[0]
Out[164]: <function __main__.fun>
In [165]: l[0]()
0
In [166]: l[1]()
1
\end{verbatim}

\end{frame} 

% ---------------------------------------------
\begin{frame}[fragile]{lambda}

{\Large``Anonymous'' functions}

\vfill
\begin{verbatim}
In [171]: f = lambda x, y: x+y

In [172]: f(2,3)
Out[172]: 5
\end{verbatim}

\vfill
{\Large Can only be an expression -- not a statement}

\end{frame} 

% ---------------------------------------------
\begin{frame}[fragile]{lambda}

{\Large Can also use keyword arguments}

\begin{verbatim}
In [186]: l = []
In [187]: for i in range(3):
    l.append(lambda x, e=i: x**e)
   .....:     
In [189]: for f in l:
    print f(3)
1
3
9
\end{verbatim}

\end{frame} 

%-------------------------------
\begin{frame}{LAB}

\begin{itemize}
  \item keyword arguments
\end{itemize}

\end{frame}


%-------------------------------
\begin{frame}{Lightning Talk}

{\center

\LARGE Lighting Talk:
\vfill
Joshua
\vfill

}
\end{frame}




\section{Lists, Tuples...}

% ---------------------------------------------
\begin{frame}[fragile]{Lists}

 {\Large List Literals}

\begin{verbatim}
>>> []
[]
>>> list()
[]
>>> [1, 2, 3]
[1, 2, 3]
>>> [1, 3.14, "abc"]
[1, 3.14, 'abc']
\end{verbatim}

\end{frame} 

% ---------------------------------------------
\begin{frame}[fragile]{Lists}

 {\Large Indexing just like all sequences}

\begin{verbatim}
>>> food = ['spam', 'eggs', 'ham']
>>> food[2]
'ham'
>>> food[0]
'spam'
>>> food[42]
Traceback (most recent call last):
  File "<stdin>", line 1, in <module>
IndexError: list index out of range
\end{verbatim}

\end{frame} 

% ---------------------------------------------
\begin{frame}[fragile]{Lists}

{\Large List are mutable}

\begin{verbatim}
>>> food = ['spam', 'eggs', 'ham']
>>> food[1] = 'raspberries'
>>> food
['spam', 'raspberries', 'ham']
\end{verbatim}

\end{frame} 

% ---------------------------------------------
\begin{frame}[fragile]{Lists}

{\Large Each element is a value, and can be in multiple lists and have multiple
names (or no name)}

\begin{verbatim}
  >>> name = 'Brian'
   >>> a = [1, 2, name]
   >>> b = [3, 4, name]
   >>> name
   'Brian'
   >>> a
   [1, 2, 'Brian']
   >>> b
   [3, 4, 'Brian']
   >>> a[2]
   'Brian'
   >>> b[2]
   'Brian'
\end{verbatim}

\end{frame} 

% ---------------------------------------------
\begin{frame}[fragile]{Lists}

{\Large \verb| .append(), .insert()|}

\begin{verbatim}
>>> food = ['spam', 'eggs', 'ham']
>>> food.append('sushi')
>>> food
['spam', 'eggs', 'ham', 'sushi']
>>> food.insert(0, 'carrots')
>>> food
['carrots', 'spam', 'eggs', 'ham', 'sushi']
\end{verbatim}

\end{frame} 

% ---------------------------------------------
\begin{frame}[fragile]{Lists}

{\large \verb| .extend()|}

\begin{verbatim}
>>> food = ['spam', 'eggs', 'ham']
>>> food.extend(['fish', 'chips'])
>>> food
['spam', 'eggs', 'ham', 'fish', 'chips']
\end{verbatim}

{\large could be any sequence:}

\begin{verbatim}
>>>  food
>>>  ['spam', 'eggs', 'ham']
>>>  silverware = ('fork', 'knife', 'spoon') # a tuple
>>>  food.extend(silverware)
>>>  food
>>>  ['spam', 'eggs', 'ham', 'fork', 'knife', 'spoon']
\end{verbatim}

\end{frame} 

% ---------------------------------------------
\begin{frame}[fragile]{Lists}

{\large \verb|pop(), remove() |}

\begin{verbatim}
In [203]: food = ['spam', 'eggs', 'ham', 'toast']
In [204]: food.pop()
Out[204]: 'toast'
In [205]: food.pop(0)
Out[205]: 'spam'
In [206]: food
Out[206]: ['eggs', 'ham']
In [207]: food.remove('ham')
In [208]: food
Out[208]: ['eggs']
\end{verbatim}

\end{frame} 


% ---------------------------------------------
\begin{frame}[fragile]{Lists}

{\large \verb|list()| accepts any sequence and returns a list of that sequence}
\begin{verbatim}
>>> word = 'Python '
>>> chars = []
>>> for char in word:
...   chars.append(char)
>>> chars
['P', 'y', 't', 'h', 'o', 'n', ' ']
>>> list(word)
['P', 'y', 't', 'h', 'o', 'n', ' ']
\end{verbatim}

\end{frame} 

% ---------------------------------------------
\begin{frame}[fragile]{Lists}

{\large If you need to change individual letters... you can do this,
but usually \verb|somestring.replace()| will be enough }

\begin{verbatim}
In [216]: name = 'Chris'
In [217]: lname = list(name)
In [218]: lname[0:2] = 'K'
In [219]: name = ''.join(lname)
In [220]: name
Out[220]: 'Kris'
\end{verbatim}

\end{frame} 

% ---------------------------------------------
\begin{frame}[fragile]{Lists}

{\large Building up strings:}

\begin{verbatim}
In [221]: msg = []

In [222]: msg.append('The first line of a message')

In [223]: msg.append('The second line of a message')

In [224]: msg.append('And one more line')

In [225]: print '\n'.join(msg)
The first line of a message
The second line of a message
And one more line
\end{verbatim}

\end{frame} 

% ---------------------------------------------
\begin{frame}[fragile]{Slicing}

{\large Slicing makes a copy}

\begin{verbatim}
In [227]: food = ['spam', 'eggs', 'ham', 'sushi']

In [228]: some_food = food[1:3]

In [229]: some_food[1] = 'bacon'

In [230]: food
Out[230]: ['spam', 'eggs', 'ham', 'sushi']

In [231]: some_food
Out[231]: ['eggs', 'bacon']
\end{verbatim}

\end{frame}


% ---------------------------------------------
\begin{frame}[fragile]{Slicing}

{\large Easy way to copy a whole list}

\begin{verbatim}
In [232]: food
Out[232]: ['spam', 'eggs', 'ham', 'sushi']

In [233]: food2 = food[:]

In [234]: food is food2
Out[234]: False

\end{verbatim}

{\Large but the copy is ``shallow''}

\end{frame} 

% ---------------------------------------------
\begin{frame}[fragile]{Slicing}

{\Large ``Shallow'' copy}

\begin{verbatim}
In [249]: food = ['spam', ['eggs', 'ham']]
In [251]: food_copy = food[:]
In [252]: food[1].pop()
Out[252]: 'ham'
In [253]: food
Out[253]: ['spam', ['eggs']]
In [256]: food.pop(0)
Out[256]: 'spam'
In [257]: food
Out[257]: [['eggs']]
In [258]: food_copy
Out[258]: ['spam', ['eggs']]
\end{verbatim}

\end{frame} 

% ---------------------------------------------
\begin{frame}[fragile]{Name Binding}

{\Large Assigning to a name does not copy:}

\begin{verbatim}
>>> food = ['spam', 'eggs', 'ham', 'sushi']
>>> food_again = food
>>> food_copy = food[:]
>>> food.remove('sushi')
>>> food
['spam', 'eggs', 'ham']
>>> food_again
['spam', 'eggs', 'ham']
>>> food_copy
['spam', 'eggs', 'ham', 'sushi']
\end{verbatim}

\end{frame} 

% ---------------------------------------------
\begin{frame}[fragile]{Iterating}

{\Large Iterating over a list}

\begin{verbatim}
>>> food = ['spam', 'eggs', 'ham', 'sushi']
>>> for x in food:
...   print x
...
spam
eggs
ham
sushi
\end{verbatim}

\end{frame} 

% ---------------------------------------------
\begin{frame}[fragile]{Processing lists}

{\Large A common pattern}

\begin{verbatim}
filtered = []
for x in somelist:
    if should_be_included(x):
        filtered.append(x)
del(somelist)  # maybe
\end{verbatim}

{\Large you don't want to be deleting items from the list while iterating...}

\end{frame} 

% ---------------------------------------------
\begin{frame}[fragile]{Mutating Lists}

{\Large if you’re going to change the list, iterate over a copy for safety }

\begin{verbatim}
>>> food = ['spam', 'eggs', 'ham', 'sushi']
>>> for x in food[:]:
   ...   # change the list somehow
   ...
\end{verbatim}

{\Large insidious bugs otherwise}

\end{frame} 

% ---------------------------------------------
\begin{frame}[fragile]{operators vs methods}

{\large What's the difference?}

\begin{verbatim}
   >>> food = ['spam', 'eggs', 'ham']
   >>> more = ['fish', 'chips']
   >>> food = food + more
   >>> food
   ['spam', 'eggs', 'ham', 'fish', 'chips']

   >>> food = ['spam', 'eggs', 'ham']
   >>> more = ['fish', 'chips']
   >>> food.extend(more)
   >>> food
   ['spam', 'eggs', 'ham', 'fish', 'chips']
\end{verbatim}

\end{frame} 

% ---------------------------------------------
\begin{frame}[fragile]{in}

\begin{verbatim}
>>> food = ['spam', 'eggs', 'ham']
>>> 'eggs' in food
True
>>> 'chicken feet' in food
False
\end{verbatim}

\end{frame} 

% ---------------------------------------------
\begin{frame}[fragile]{reverse()}

{\large Iterating over a list}

\begin{verbatim}
>>> food = ['spam', 'eggs', 'ham']
>>> food.reverse()
>>> food
['ham', 'eggs', 'spam']
\end{verbatim}

\end{frame} 

% ---------------------------------------------
\begin{frame}[fragile]{sort()}

\vfill
\begin{verbatim}
>>> food = ['spam', 'eggs', 'ham', 'sushi']
>>> food.sort()
>>> food
['eggs', 'ham', 'spam', 'sushi']
\end{verbatim}

\vfill
{\Large note:}

\vfill
\begin{verbatim}
>>> food = ['spam', 'eggs', 'ham', 'sushi']
>>> result = food.sort()
>>> print result
None
\end{verbatim}

\vfill
\end{frame} 

% ---------------------------------------------
\begin{frame}[fragile]{Sorting}

{\large How should this sort?}

\begin{verbatim}
>>> s
[[2, 'a'], [1, 'b'], [1, 'c'], [1, 'a'], [2, 'c']]
\end{verbatim}

\pause

\begin{verbatim}
>>> s.sort()
>>> s
[[1, 'a'], [1, 'b'], [1, 'c'], [2, 'a'], [2, 'c']]
\end{verbatim}

\end{frame} 

% ---------------------------------------------
\begin{frame}[fragile]{Sorting}

{\large You can specify your own compare function:}

\begin{verbatim}
In [279]: s = [[2, 'a'], [1, 'b'], [1, 'c'], [1, 'a'], [2, 'c']]
In [281]: def comp(s1,s2):
   .....:     if s1[1] > s2[1]: return 1
   .....:     elif s1[1]<s2[1]: return -1
   .....:     else:
   .....:         if s1[0] > s2[0]: return 1
   .....:         elif s1[0] < s2[0]: return -1
   .....:     return 0
In [282]: s.sort(comp)
In [283]: s
Out[283]: [[1, 'a'], [2, 'a'], [1, 'b'], [1, 'c'], [2, 'c']]
\end{verbatim}
\end{frame} 

% ---------------------------------------------
\begin{frame}[fragile]{Sorting}

{\Large
Mixed types can be sorted.
}
\vfill
{\center \Large

``objects of different types always compare unequal, and are ordered
consistently but arbitrarily.''

}

\vfill
\url{http://docs.python.org/reference/expressions.html#not-in}
\end{frame} 

% ---------------------------------------------
\begin{frame}[fragile]{Searching}

{\Large Finding or Counting items}

\begin{verbatim}
In [288]: l = [3,1,7,5,4,3]

In [289]: l.index(5)
Out[289]: 3

In [290]: l.count(3)
Out[290]: 2
\end{verbatim}

\end{frame} 


% ---------------------------------------------
\begin{frame}[fragile]{List Performance }

\begin{itemize}
  \itemindexing is fast and constant time: O(1)
  \item x in s proportional to n: O(n)
  \item visiting all is proportional to n: O(n)
  \item operating on the end of list is fast and constant time: O(1) \\
     append(), pop()
  \item operating on the front (or middle) of the list depends on n: O(n)\\
     pop(0), insert(0, v) \\
     But, reversing is fast. Also, collections.deque
\end{itemize}

\url{ http://wiki.python.org/moin/TimeComplexity}

\end{frame} 


% ---------------------------------------------
\begin{frame}[fragile]{Lists vs. Tuples}

\vfill
{\Large List or Tuples}

\vfill
{\Large
If it needs to mutable: list

\vfill
If it needs to be immutable: tuple\\
}
\hspace{0.2in}{\large (dict key, safety when passing to a function) }

\vfill
{\Large Otherwise ... taste and convention}

\end{frame} 

% ---------------------------------------------
\begin{frame}[fragile]{List vs Tuple}

\vfill
{\LARGE Convention:}

\vfill
{\Large Lists are Collections (homogeneous):\\[0.1in]
-- contain values of the same type \\ 
-- simplifies iterating, sorting, etc
}

\vfill
{\Large tuples are mixed types:\\[0.1in]
-- Group multiple values into one logical thing
-- Kind of like simple C structs.
}
\vfill

\end{frame} 

% ---------------------------------------------
\begin{frame}[fragile]{List vs Tuple}

{\Large
\begin{itemize}
  \item Do the same operation to each element?
  \item Small collection of values which make a single logical item?
  \item To document that these values won't change?
  \item Build it iteratively?
  \item Transform, filter, etc?
\end{itemize}
}

\end{frame} 

% ---------------------------------------------
\begin{frame}[fragile]{List vs Tuple}

{\Large
\begin{itemize}
  \item Do the same operation to each element? {\bf list}
  \item Small collection of values which make a single logical item? {\bf tuple}
  \item To document that these values won't change? {\bf tuple}
  \item Build it iteratively? {\bf list}
  \item Transform, filter, etc? {\bf list}
\end{itemize}
}

\end{frame} 

% ----------------------------------------------
\begin{frame}[fragile]{Named Tuple (Collections Module) }

\begin{verbatim}
>>> Point = collections.namedtuple('Point',('x','y'))
>>> p = Point(3.4, 5.2)
>>> p
 Point(x=3.4, y=5.2)
>>> p.x
 3.4
>>> p[1]
 5.2
>>> p = Point(y=2.3, x=3.1)
>>>  p
 Point(x=3.1, y=2.3)
\end{verbatim}

\end{frame} 

%---------------------------------------------
\begin{frame}[fragile]{Named Tuple (Collections Module) }

\vfill
{\Large Named Tuple}

\vfill
{\Large handy for database records: sqlite, csv, etc}

\vfill
\url{http://docs.python.org/library/collections.html#module-collections}

\end{frame} 

% ----------------------------------------------
\begin{frame}[fragile]{List comprehensions}

{\Large A bit of functional programming:}

\begin{verbatim}
new_list = [expression for variable in a_list]
\end{verbatim}

{\Large same as for loop:}

\begin{verbatim}
new_list = []
for variable in a_list:
    new_list.append(expression)
\end{verbatim}

\end{frame} 

% ----------------------------------------------
\begin{frame}[fragile]{List comprehensions}

{\Large Examples:}

\begin{verbatim}
In [341]: [x**2 for x in range(3)]
Out[341]: [0, 1, 4]

In [342]: [x+y for x in range(3) for y in range(2)]
Out[342]: [0, 1, 1, 2, 2, 3]

In [343]: [x*2 for x in range(6) if not x%2]
Out[343]: [0, 4, 8]
\end{verbatim}

\end{frame} 

% ----------------------------------------------
\begin{frame}[fragile]{List comprehensions}

{\Large Remember this from last week?}

\begin{verbatim}
[name for name in dir(__builtin__) if "Error" in name]

['ArithmeticError',
 'AssertionError',
 'AttributeError',
 'BufferError',
 'EOFError',
 'EnvironmentError',
\end{verbatim}

\end{frame} 

% ----------------------------------------------
\begin{frame}[fragile]{Generator Expressions}

{\Large Like a list comprehension, but generates the items on the fly:}

\begin{verbatim}
In [393]: g = ( x**2 for x in [3, 4, 5])

In [394]: g
Out[394]: <generator object <genexpr> at 0x17b0df0>

In [395]: for i in g:
    print i
   .....:     
9
16
25
\end{verbatim}

\end{frame} 



\begin{frame}[fragile]{List Docs}

\vfill
{\Large The list docs:}

\vfill
\url{http://docs.python.org/library/stdtypes.html#mutable-sequence-types}

\vfill
(actually any mutable sequence....)

\end{frame} 



%-------------------------------
\begin{frame}{LAB}

\vfill
{\LARGE Dan's list Lab}
\vfill

\end{frame}

%-------------------------------
\begin{frame}{Lightning Talk}

{\center

\LARGE Lighting Talk:
\vfill
David
\vfill

}
\end{frame}

% ##################################
\section{Dictionaries and Sets}

% ---------------------------------------------
\begin{frame}[fragile]{Dictionary}

{\Large Python calls it a \verb|dict| }

\vfill
{\Large Other terms:}
\begin{itemize}
  \item dictionary
  \item associative array
  \item map
  \item hash table
  \item hash
  \item key-value pair
\end{itemize}

\vfill

\end{frame} 

% ---------------------------------------------
\begin{frame}[fragile]{Dictionary Constructors}

\begin{verbatim}
>>> {'key1': 3, 'key2': 5}
{'key1': 3, 'key2': 5}

>>> dict([('key1', 3),('key2', 5)])
{'key1': 3, 'key2': 5}

>>> dict(key1=3, key2= 5)
{'key1': 3, 'key2': 5}

>>> d = {}
>>> d['key1'] = 3
>>> d['key2'] = 5
>>> d
{'key1': 3, 'key2': 5}
\end{verbatim}
% {\Large Which to use depends on the shape of your data}

\end{frame} 

% ---------------------------------------------
\begin{frame}[fragile]{Dictionary Indexing}

\begin{verbatim}
>>> d = {'name': 'Brian', 'score': 42}
>>> d['score']
42
>>> d = {1: 'one', 0: 'zero'}
>>> d[0]
'zero'
>>> d['non-­‐existing key']
Traceback (most recent call last):
  File "<stdin>", line 1, in <module>
KeyError: 'fish'
\end{verbatim}

\end{frame} 

% ---------------------------------------------
\begin{frame}[fragile]{Dictionary Indexing}

{\Large Keys can be any immutable:}
\begin{itemize}
  \item numbers
  \item string
  \item tuples
\end{itemize}

\begin{verbatim}
In [325]: d[3] = 'string'
In [326]: d[3.14] = 'pi'
In [327]: d['pi'] = 3.14
In [328]: d[ (1,2,3) ] = 'a tuple key'
In [329]: d[ [1,2,3] ] = 'a list key'
   TypeError: unhashable type: 'list'
\end{verbatim}

\vfill
Actually -- any "hashable" type.
\end{frame} 

% ---------------------------------------------
\begin{frame}[fragile]{Dictionary Indexing}

\vfill
{\Large hash functions convert arbitrarily large data to a small proxy (usually int)

\vfill
always return the same proxy for the same input

\vfill
MD5, SHA, etc
\vfill
}
\end{frame} 

% ---------------------------------------------
\begin{frame}[fragile]{Dictionary Indexing}

\vfill
{\Large
Dictionaries hash the key to an integer proxy and use it to find the key and value
}
\vfill
{\Large
Key lookup is efficient because the hash function leads directly to a bucket with a very few keys (often just one)
}
\vfill
\end{frame} 

% ---------------------------------------------
\begin{frame}[fragile]{Dictionary Indexing}

\vfill
{\Large
What would happen if the proxy changed after storing a key?
}
\vfill
{\Large
Hashability requires immutability}
\vfill
\end{frame} 

% ---------------------------------------------
\begin{frame}[fragile]{Dictionary Indexing}

\vfill
{\Large

Key lookup is very efficient

\vfill
Same average time regardless of size
}

\vfill
also... Python name look-ups are implemented with dict:

 --- it’s highly optimized
\end{frame} 


% ---------------------------------------------
\begin{frame}[fragile]{Dictionary Indexing}

\vfill
{\Large
{\center 

key to value

lookup is one way

}}
\vfill
{\Large
{\center 

value to key

requires visiting the whole dict

}}

\vfill
{\Large
if you need to check dict values often, create another dict or set (up to you to keep them in sync)

}
\vfill
\end{frame} 

% ---------------------------------------------
\begin{frame}[fragile]{Dictionary Indexing}

\vfill
{\Large
dictionaries have no defined order
}
\vfill
\begin{verbatim}
In [352]: d = {'one':1, 'two':2, 'three':3}

In [353]: d
Out[353]: {'one': 1, 'three': 3, 'two': 2}

In [354]: d.keys()
Out[354]: ['three', 'two', 'one']
\end{verbatim}
\vfill
\end{frame} 

%-------------------------------
\begin{frame}[fragile]{dict iterating}

{\Large \verb|for| iterates the keys}
\vfill
\begin{verbatim}
>>> d = {'name': 'Brian', 'score': 42}
>>> for x in d:
...   print x
...
score name
\end{verbatim}
\vfill
{note the different order...}
\end{frame}

%-------------------------------
\begin{frame}[fragile]{dict keys and values}

\vfill
\begin{verbatim}
>>> d.keys()
['score', 'name']

>>> d.values()
[42, 'Brian']

>>> d.items()
[('score', 42), ('name', 'Brian')]
\end{verbatim}
\vfill
\end{frame}

%-------------------------------
\begin{frame}[fragile]{dict keys and values}

{\Large iterating on everything}
\vfill
\begin{verbatim}
>>> d = {'name': 'Brian', 'score': 42}
>>> for k, v in d.items():
...   print "%s: %s" % (k, v)
...
score: 42
name: Brian
\end{verbatim}
\vfill
\end{frame}

% ---------------------------------------------
\begin{frame}[fragile]{Dictionary Performance }

\begin{itemize}
  \itemindexing is fast and constant time: O(1)
  \item x in s cpnstant time: O(1)
  \item visiting all is proportional to n: O(n)
  \item inserting is constant time: O(1)
  \item deleting is constant time: O(1)
\end{itemize}

\vfill
\url{ http://wiki.python.org/moin/TimeComplexity}

\end{frame} 

% ----------------------------------------------
\begin{frame}[fragile]{Dict Comprehensions}

{\Large You can do it with dicts, too:}

\begin{verbatim}
new_dict = { key:value for variable in a_sequence}
\end{verbatim}

{\Large same as for loop:}

\begin{verbatim}
new_dict = {}
for key in a_list:
    new_dict[key] = value
\end{verbatim}

\end{frame} 

% ----------------------------------------------
\begin{frame}[fragile]{Dict Comprehensions}

{\Large Example}

\begin{verbatim}
In [340]: { i: "this_%i"%i for i in range(5) }
Out[340]: {0: 'this_0', 1: 'this_1', 2: 'this_2',
           3: 'this_3', 4: 'this_4'}
\end{verbatim}

\vfill
(not as useful with the dict() constructor...)
\end{frame} 


% ---------------------------------------------
\begin{frame}[fragile]{ Sets }

\vfill
{\Large \verb|set| is an unordered collection of distinct values}

\vfill
{\Large Essentially a dict with only keys}

\vfill

\end{frame} 

%-------------------------------
\begin{frame}[fragile]{Set Constructors}

\vfill
\begin{verbatim}
>>> set()
set([])
>>> set([1, 2, 3])
set([1, 2, 3])
# as of 2.7
>>> {1, 2, 3}
set([1, 2, 3])
>>> s = set()
>>> s.update([1, 2, 3])
>>> s
set([1, 2, 3])
\end{verbatim}
\vfill

\end{frame}

% ---------------------------------------------
\begin{frame}[fragile]{ Set Properties}

\vfill
{\Large \verb|Set| members must be hashable}

\vfill
{\Large Like dictionary keys -- and for same reason (efficient lookup)}

\vfill
{\Large No indexing (unordered) }

\vfill
\begin{verbatim}
>>> s[1]
Traceback (most recent call last):
  File "<stdin>", line 1, in <module>
TypeError: 'set' object does not support indexing
\end{verbatim}

\vfill
\end{frame} 

% ---------------------------------------------
\begin{frame}[fragile]{ Set Methods}

\begin{verbatim}
>> s = set([1])
>>> s.pop() # an arbitrary member
1
>>> s.pop()
Traceback (most recent call last):
  File "<stdin>", line 1, in <module>
KeyError: 'pop from an empty set'

>>> s = set([1, 2, 3])
>>> s.remove(2)
>>> s.remove(2)
Traceback (most recent call last):
  File "<stdin>", line 1, in <module>
KeyError: 2
\end{verbatim}

\vfill
\end{frame} 

% ---------------------------------------------
\begin{frame}[fragile]{ Set Methods}

\begin{verbatim}
s.isdisjoint(other)

s.issubset(other)

s.union(other, ...)

s.intersection(other, ...)

s.difference(other, ...)

s.symmetric_difference( other, ...)
\end{verbatim}

\vfill
\end{frame} 

% ---------------------------------------------
\begin{frame}[fragile]{ Frozen Set}

\vfill
{\Large Also \verb|frozenset|}

\vfill
{\Large immutable -- for use as a key in a dict\\
(or another set...)}

\vfill
\begin{verbatim}
>>> fs = frozenset((3,8,5))
>>> fs.add(9)
Traceback (most recent call last):
  File "<stdin>", line 1, in <module>
AttributeError: 'frozenset' object has no attribute 'add'
\end{verbatim}

\vfill
\end{frame} 

% ---------------------------------------------
\begin{frame}[fragile]{Function arguments in variables}

{\Large function arguments are really just\\
 -- a tuple (positional arguments) \\
 -- a dict (keyword arguments) \\
}
\begin{verbatim}
def f(x, y, w=0, h=0):
    print "position: %s, %s -- shape: %s, %s"%(x, y, w, h)

position = (3,4)
size = {'h': 10, 'w': 20}

>>> f( *position, **size)
position: 3, 4 -- shape: 20, 10
\end{verbatim}

\end{frame} 

% ---------------------------------------------
\begin{frame}[fragile]{Function parameters in variables}

{\Large You can also pull in the parameters out in the function as a tuple and a dict
}
\begin{verbatim}
def f(*args, **kwargs):
    print "the positional arguments are:", args
    print "the keyword arguments are:", kwargs
 
In [389]: f(2, 3, this=5, that=7)
the positional arguments are: (2, 3)
the keyword arguments are: {'this': 5, 'that': 7}
\end{verbatim}

\end{frame} 





%-------------------------------
\begin{frame}{LAB}

{\LARGE Dan's dict LAB

\vfill
or

\vfill
Optional LAB
}

\begin{itemize}
  \item Coding Kata 14 - Dave Thomas \\
    \url{http://codekata.pragprog.com/2007/01/ kata_fourteen_t.html}
  \item See how far you can get on this task using “The Adventures of Sherlock Holmes” as input: sherlock.txt in the week04 directory (UTF-8)
  \item  This is intentionally open-ended and underspecified. There are many interesting decisions to make.
\end{itemize}

\end{frame}

%%-------------------------------
%\begin{frame}{Lightning Talk}
%
%{\center
%
%\LARGE Lighting Talk:
%\vfill
%Rob
%\vfill
% 

\end{document}
